\documentclass[12pt]{report}

\usepackage[utf8]{inputenc}
\usepackage{graphicx}
\usepackage[T1]{fontenc}
\usepackage{lipsum}
\usepackage[hidelinks]{hyperref}
\usepackage{textcomp}
\usepackage{color}
\usepackage{gensymb}
\usepackage{amsmath} % \text
\usepackage{listings} % \lstlisting
\usepackage{float} % \begin{figure}[H]
\usepackage{geometry}
%\usepackage{mathpazo} % Palatino font

\graphicspath{{images/}}

%%%%%%%%%%%%%%%%%%%%%%%%%%%
\makeatletter
\def\@makechapterhead#1{%
  \vspace*{50\p@}%
  {\parindent \z@ \raggedright \normalfont
    \interlinepenalty\@M
    \Huge \bfseries #1\par\nobreak
    \vskip 40\p@
  }}
\def\@makeschapterhead#1{%
  \vspace*{50\p@}%
  {\parindent \z@ \raggedright
    \normalfont
    \interlinepenalty\@M
    \Huge  \bfseries  #1\par\nobreak
    \vskip 40\p@
  }}
\makeatother
%%%%%%%%%%%%%%%%%%%%%%%%%%%
% Kod iznad preuzet (uz manje modifikacije) sa: http://tex.stackexchange.com/a/120743
% Bez ovog koda iznad svakog poglavlja bi pisalo 'Chapter 1', 'Chapter 2', itd.

\newcommand{\ch}{č}
\newcommand{\CH}{Č}
\newcommand{\cj}{ć}
\newcommand{\CJ}{Ć}
\newcommand{\sh}{š}
\newcommand{\SH}{Š}
\newcommand{\zh}{ž}
\newcommand{\ZH}{Ž}
% Radi brzeg pisanja afrikata na engleskoj tastaturi (\dj i \DJ su vec definisani)

\newcommand{\midtilde}{\raisebox{0.6ex}{\texttildelow}} % Iz nekog razloga nije bas jednostavno napisati finu tildu u LaTeX-u, ali ovo radi ok
\renewcommand{\contentsname}{Sadr\zh aj}
\renewcommand\bibname{Literatura}
\renewcommand{\figurename}{Slika}

\definecolor{dkgreen}{rgb}{0,0.6,0}
\lstset{frame=tb,
  language=C++,
  aboveskip=3mm,
  belowskip=3mm,
  showstringspaces=false,
  columns=flexible,
  basicstyle={\small\ttfamily},
  numbers=none,
  numberstyle=\tiny\color{gray},
  keywordstyle=\color{blue},
  commentstyle=\color{dkgreen},
  stringstyle=\color{red},
  breaklines=true,
  breakatwhitespace=true,
  tabsize=3,
  basicstyle=\ttfamily
} %https://stackoverflow.com/a/3175141/7971194

\begin{document}

\section*{Sa\zh etak}
U radu su obja\sh njeni klju\ch ni pojmovi vezani za interpolaciju video sekvenci, poput frejmova, vektora pomaka i opti\ch kog toka. Nakon toga,
obja\sh njen je na\ch in na koji mo\zh emo interpolirati frejmove ako je opti\ch ki tok poznat, te problemi koji nastaju i kako ih rije\sh iti. U glavnom dijelu rada su zatim
obja\sh njeni algoritmi pronala\zh enja opti\ch kog toka, koji mogu raditi na principu uparivanja blokova, fazne korelacije ili putem metoda zasnovanim na diferencijalnim
jedna\ch inama. Prostom primjenom ovih algoritama \cj e interpolirani frejmovi biti veoma slabe kvalitete, tako da su tako\dj er opisane metode za popravljanje gre\sh aka
i artefakta. U posljednjem poglavlju je opisana implementacija interpolatora u programskom jeziku C++ uz kori\sh tenje OpenCV biblioteke.

\section*{Abstract}
This work explains some key terms related to video sequence interpolation, such as frames, motion vectors, and optical flow. After that, we explain how one might
interpolate frames if the optical flow is known, the problems that arise from that approach, and how to solve them. The main part of the work deals with optical flow
algorithms, categorized into block matching, phase correlation, or differential equation based methods. Basic usage of these algorithms generates frames
of poor quality, hence we explain some methods used to correct errors and artifacts that arise during the process. The final chapter explains the implementation
of an interpolator program, written in the C++ programming language, and using the OpenCV library.

\newpage
\section*{Izjava autora}
Ja dolje potpisani izjavljujem da ovaj podnesak predstavlja moj vlastiti rad i da, prema mom
saznanju i vjerovannu, u njemu nije sadr\zh an bilo koji dio koji je prethodno objavljen ili napisan
od strane drugih osoba, niti materijal koji je prihva\cj en u svrhu dodjele diplome bilo kojeg
stepena na bilo kojoj visoko\sh kolskoj ili nau\ch noj instituciji na svijetu. Ipak, obzirom na ozbiljnost
izrade jednog nau\ch nog rada, autor zadr\zh ava pravo da radi komparacije koristi i izvore
koji nisu njegovo originalno djelo, ali su obavezno navedeni u tekstu i referencama rada.

Autorska prava diplomskog rada ostaju pri autoru rada i Elektrotehni\ch kom fakultetu Univerziteta
u Sarajevu.

Ova izjava je napisana u skladu sa Pravilnikom o strukturi i sadr\zh aju doktorske disertacije i 
magistarskog rada na Elektrotehni\ch kom fakultetu u Sarajevu (br. 04-1-673/11, dana 17.01.2011.
godine).
\vfill
\begin{center}
\noindent\rule{8cm}{0.4pt}\\
Muhamed Pari\cj\ 
\vfill
U Sarajevu,\\
septembar 2017.
\end{center}

\tableofcontents

\chapter{Uvod}
\section{Op\cj enito o video sekvencama}
Radi jednostvnosti, mi \cj emo imati relativno jednostavan pogled na video sekvence. Fokus rada je na direktnoj manipulaciji nad frejmovima (sli\ch icama) i pikselima radi dobivanja krajnje sekvence, 
ne u algoritmima dekodiranja i enkodiranja video fajlova, koji mogu biti veoma slo\zh eni. Konkretno, mi \cj emo na video sekvencu gledati kao obi\ch an niz frejmova. Svaki frejm je matrica piksela. 
Veli\ch ina frejma naravno zavisi od veli\ch ine konkretnog fajla na kojem radimo, ali za primjere \cj emo koristiti frejmove dimenzija 1280x720 piksela (1280 piksela \sh irine i 720 piksela visine).
 Video sekvenc  ove dimenzije se \ch esto zovu HD (\textit{High Definition}) video sekvence. Neke druge \ch esto kori\sh tene dimenzije uklju\ch uju 640x360, 1920x1080 (\textit{Full HD}), 2560x1440 
 (\textit{2K}), 3840x2160(\textit{4K}), itd. HD veli\ch ina je tako\dj er pogodno jer jedan frejm sadr\zh i \midtilde 1 milion piksela (ta\ch no 921600), \sh to prora\ch une \ch ija preciznost nije od izrazite 
 va\zh nosti \ch ini lak\sh im. 
 
 Jo\sh\ jedna va\zh na osobina video sekvenci jeste \textit{framerate}, broj frejmova u sekundi. Framerate video sekvence zavisi od primjene. Za filmove, standardni framerate
iznosi 24 \textit{fps} (\textit{frames per second}, broj frejmova u sekundi). Video igre naj\ch e\sh \cj e prikazuju 30 ili 60 fps. Ve\cj ina monitora ne mo\zh e prikazivati vi\sh e od 60 fps. Iako predstavlja
va\zh nu osobinu video sekvenci, nama framerate nije od naro\ch ite va\zh nosti. Ulaz i izlaz predstavljaju sekvence frejmova, koje mogu biti prikazane proizvoljno \ch esto. O tome koji framerate
je pogodan za koju namjenu \cj e biti vi\sh e govora u dijelu o primjenama video interpolacije

Zavisno od toga u kojoj smo fazi obrade frejma, piksel predstavlja jednu od dvije razli\ch ite stvari. Po\ch etni i krajnji video fajlovi imaju piksele sa 3 komponente: Crvena, zelena i plava. Svaka 
komponenta ima cjelobrojnu ja\ch inu izme\dj u 0 i 255, uklju\ch ivo. Me\dj utim, u fazi obrade, \ch esto je pogodnije gledati na svaki piksel kao samo jedan broj (u istom opsegu) koji
predstavlja ja\ch inu svjetlosti piksela (drugim rije\ch ima, frejm postaje crno-bijel). Intenzitet svjetlosti svakog piksela \cj e biti obi\ch na aritmeti\ch ka sredina crvene, zelene i plave komponente.

\subsection{Video fajlovi}
Iako fokus rada nije na video fajlovima, vrijedi spomenuti nekoliko detalja. Proces \ch itanja i pisanja video sekvenci u fajlove \cj e obavljati OpenCV biblioteka (o kojoj \cj e biti vi\sh e rije\ch i u 
posljednjem poglavlju). \CH uvati video fajlove kao obi\ch ne nizove frejmova bi bilo izuzetno neefikasno. Naime, video HD dimenzije du\zh ine 2 sata (pri \ch emu svaki piksel zauzima 3 bajta i
framerate iznosi 24 fps) bez ikakve kompresije bi zauzimao $2 * 3600 * 24 * 1280 * 720 * 3 = 477757440000$ bajta (\midtilde 466 GB), gdje u stvarnosti ta veli\ch ina iznosi naj\ch e\sh \cj e 
nekoliko gigabajta. Kompresija video fajlova se zasniva na 2 osnovna principa: \textit{intraframe} kompresija i \textit{interframe} kompresija. Intraframe kompresija se bavi smanjivanjem veli\ch ine
svakog pojedina\ch nog frejma, \sh to nema dodirnih ta\ch aka sa tehnikama interpolacije frejmova. Interframe kompresija poku\sh ava smanjiti veli\ch inu krajnjeg fajla tako \sh to tra\zh i sli\ch nosti
izme\dj u uzastopnih frejmova, te umjesto spa\sh avanja \ch itavih frejmova, spasi razlike izme\dj u trenutnog i pro\sh log frejma (ili trenutnog i narednog). Poglavlje 3 je posve\cj eno upravo tehnikama
koje tra\zh e sli\ch nosti izme\dj u frejmova. Osim video sadr\zh aja, video fajlovi sadr\zh e i druge komponente poput jedne ili vi\sh e audio komponente, prijevoda, menija, itd. Ove komponente
ne\cj e biti obra\dj ivane u ovom radu, niti ih OpenCV biblioteka podr\zh ava.

\section{Primjene video interpolacije}

\chapter{Osnovne tehnike}
\input{"chapters/Osnovne tehnike"}

\chapter{Opti\ch ki tok}
\input{"chapters/Opticki tok"}

\chapter{Uklanjanje gre\sh aka}
\input{"chapters/Uklanjanje gresaka"}

\chapter{Implementacija interpolatora kori\sh tenjem OpenCV biblioteke}
\input{"chapters/Implementacija interpolatora"}

\chapter{Zaklju\ch ak}
Interpolacija frejmova ima mnogobrojne realne primjene, od pobolj\sh anja kvalitete video zapisa, do video kompresije. Algoritmi i metode kori\sh teni za interpolaciju
tako\dj er imaju mnoge druge primjene. Opti\ch ki tok, prividno kretanje objekata kroz video zapis je od klju\ch ne va\zh nosti za mnoge oblasti ra\ch unarske vizije, poput
raznih vrsta robota, kamera koje detektuju, prate i mjere pokret, te prepoznavanja i pra\cj enja raznih oblika. U svrhu interpolacije smo objasnili nekoliko algoritama, od kojih
neki nisu uop\sh te zasnovani na izra\ch unavanju opti\ch kog toka, a drugi koriste slo\zh ene matemati\ch ke metode za precizno i efikasno ra\ch unanje. Neke metode,
poput obi\ch ne linearne interpolacije, su toliko brze da mogu biti kori\sh tene u realnom vremenu. Kao i u svim ostalim oblastima ra\ch unarskih nauka, jo\sh\ uvijek se radi
na tra\zh enju boljih, br\zh ih i efikasnijih metoda.

Od metoda koje izra\ch unaju opti\ch ki tok, najjednostavnije su zasnovane na \ch istom uparivanju blokova piksela i tra\zh enju blokova koji se najbolje podudaraju.
S obzirom da bi izra\ch unavanje kvalitete uparivanja svaka dva para blokova piksela bilo neprihvatljivo sporo, pretraga je ograni\ch ena na relativno mali broj potencijalnih
kandidata, i ubrzana heuristi\ch kim metodama koje se razlikuju od algoritma do algoritma.

Naprednije metode uklju\ch uju uparivanje ubrzano kori\sh tenjem brze Fourierove transformacije, rje\sh avanjem sistema linearnih jedna\ch ine, ili iterativnim pobolj\sh anjem
kvalitete opti\ch kog toka. Nakon izra\ch unavanja opti\ch kog toka, ideja je uvijek ista: piksele pomjerimo jedan dio puta du\zh\ njihovog vektora pomaka, ostale piksele izra\ch unamo
drugim metodama (poput linearne interpolacije), te popravimo koliko mo\zh emo nastale artefakte. Kvalitet generisanih frejmova nikada ne\cj e biti isti kao kvalitet originalnih,
ali kori\sh tenjem boljih i br\zh ih metoda, pribli\zh avamo se tom idealnom cilju.

\begin{thebibliography}{30}
\bibitem{blockmatching}
Barjatya, A, "Block Matching Algorithms for Motion Estimation"
\bibitem{lowcomplexity}
Zhai, J., Keman, Y., Li, J., Shipeng, L., "A Low Complexity Motion Compensated Frame Interpolation Method"
\bibitem{phaseonly}
Reyes, A., Alba, A., Arce-Santana, E. R., "Optical Flow Estimation using Phase Only-Correlation"
\bibitem{matlablk}
\url{https://www.mathworks.com/help/vision/ref/opticalflowlk-class.html}
\bibitem{matlabhs}
\url{https://www.mathworks.com/help/vision/ref/opticalflowhs-class.html}
\bibitem{mefi}
Chetty, S., MacArthur, R., Wood, S., "Motion Estimated Frame Interpolation"
\bibitem{opencvabout}
\url{http://opencv.org/about.html}
\bibitem{qtabout}
\url{https://wiki.qt.io/About_Qt}
\bibitem{wikivideo}
\url{https://en.wikipedia.org/wiki/Video}
\bibitem{implementation}
\url{https://github.com/muhamedparic/Interpolator/tree/master/Implementation/Interpolator}
\bibitem{lknutshell}
Rojas, R., "Lucas-Kanade ina Nutshell"
\bibitem{mcfiwiki}
\url{https://en.wikipedia.org/wiki/Motion_interpolation}
\bibitem{butterflow}
\url{https://github.com/dthpham/butterflow}
\bibitem{wikioptflow}
\url{https://en.wikipedia.org/wiki/Optical_flow}
\bibitem{convolution}
\url{https://docs.gimp.org/en/plug-in-convmatrix.html}
\bibitem{wikikernel}
\url{https://en.wikipedia.org/wiki/Kernel_(image_processing)}
\bibitem{h265}
\url{http://x265.org/hevc-h265/}
\bibitem{opencvshitomasi}
\url{http://docs.opencv.org/3.0-beta/doc/py_tutorials/py_feature2d/py_shi_tomasi/py_shi_tomasi.html}
\bibitem{gaussian}
\url{https://homepages.inf.ed.ac.uk/rbf/HIPR2/gsmooth.htm}
\bibitem{gausscalc}
\url{http://dev.theomader.com/gaussian-kernel-calculator/}
\bibitem{cvdocs}
\url{http://docs.opencv.org/3.2.0/}
\bibitem{examplevideo}
5secondfilms, "Missing", 08.06.2011., YouTube, \url{https://www.youtube.com/watch?v=UiyDmqO59QE}
\end{thebibliography}
\end{document}