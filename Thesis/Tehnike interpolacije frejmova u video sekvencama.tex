\documentclass[12pt]{report}

\usepackage[utf8]{inputenc}
\usepackage{graphicx}
\usepackage[T1]{fontenc}
\usepackage{lipsum}
\usepackage[hidelinks]{hyperref}
\usepackage{textcomp}
\usepackage{color}
\usepackage{gensymb}
\usepackage{amsmath} % \text
\usepackage{listings} % \lstlisting
\usepackage{float} % \begin{figure}[H]
\usepackage{geometry}
%\usepackage{mathpazo} % Palatino font

\graphicspath{{images/}}

%%%%%%%%%%%%%%%%%%%%%%%%%%%
\makeatletter
\def\@makechapterhead#1{%
  \vspace*{50\p@}%
  {\parindent \z@ \raggedright \normalfont
    \interlinepenalty\@M
    \Huge \bfseries #1\par\nobreak
    \vskip 40\p@
  }}
\def\@makeschapterhead#1{%
  \vspace*{50\p@}%
  {\parindent \z@ \raggedright
    \normalfont
    \interlinepenalty\@M
    \Huge  \bfseries  #1\par\nobreak
    \vskip 40\p@
  }}
\makeatother
%%%%%%%%%%%%%%%%%%%%%%%%%%%
% Kod iznad preuzet (uz manje modifikacije) sa: http://tex.stackexchange.com/a/120743
% Bez ovog koda iznad svakog poglavlja bi pisalo 'Chapter 1', 'Chapter 2', itd.

\newcommand{\ch}{č}
\newcommand{\CH}{Č}
\newcommand{\cj}{ć}
\newcommand{\CJ}{Ć}
\newcommand{\sh}{š}
\newcommand{\SH}{Š}
\newcommand{\zh}{ž}
\newcommand{\ZH}{Ž}
% Radi brzeg pisanja afrikata na engleskoj tastaturi (\dj i \DJ su vec definisani)

\newcommand{\midtilde}{\raisebox{0.6ex}{\texttildelow}} % Iz nekog razloga nije bas jednostavno napisati finu tildu u LaTeX-u, ali ovo radi ok
\renewcommand{\contentsname}{Sadr\zh aj}
\renewcommand\bibname{Literatura}
\renewcommand{\figurename}{Slika}

\definecolor{dkgreen}{rgb}{0,0.6,0}
\lstset{frame=tb,
  language=C++,
  aboveskip=3mm,
  belowskip=3mm,
  showstringspaces=false,
  columns=flexible,
  basicstyle={\small\ttfamily},
  numbers=none,
  numberstyle=\tiny\color{gray},
  keywordstyle=\color{blue},
  commentstyle=\color{dkgreen},
  stringstyle=\color{red},
  breaklines=true,
  breakatwhitespace=true,
  tabsize=3,
  basicstyle=\ttfamily
} %https://stackoverflow.com/a/3175141/7971194

\begin{document}

\section*{Sa\zh etak}
U radu su obja\sh njeni klju\ch ni pojmovi vezani za interpolaciju video sekvenci, poput frejmova, vektora pomaka i opti\ch kog toka. Nakon toga,
obja\sh njen je na\ch in na koji mo\zh emo interpolirati frejmove ako je opti\ch ki tok poznat, te problemi koji nastaju i kako ih rije\sh iti. U glavnom dijelu rada su zatim
obja\sh njeni algoritmi pronala\zh enja opti\ch kog toka, koji mogu raditi na principu uparivanja blokova, fazne korelacije ili putem metoda zasnovanim na diferencijalnim
jedna\ch inama. Prostom primjenom ovih algoritama \cj e interpolirani frejmovi biti veoma slabe kvalitete, tako da su tako\dj er opisane metode za popravljanje gre\sh aka
i artefakta. U posljednjem poglavlju je opisana implementacija interpolatora u programskom jeziku C++ uz kori\sh tenje OpenCV biblioteke.

\section*{Abstract}
This work explains some key terms related to video sequence interpolation, such as frames, motion vectors, and optical flow. After that, we explain how one might
interpolate frames if the optical flow is known, the problems that arise from that approach, and how to solve them. The main part of the work deals with optical flow
algorithms, categorized into block matching, phase correlation, or differential equation based methods. Basic usage of these algorithms generates frames
of poor quality, hence we explain some methods used to correct errors and artifacts that arise during the process. The final chapter explains the implementation
of an interpolator program, written in the C++ programming language, and using the OpenCV library.

\newpage
\section*{Izjava autora}
Ja dolje potpisani izjavljujem da ovaj podnesak predstavlja moj vlastiti rad i da, prema mom
saznanju i vjerovanju, u njemu nije sadr\zh an bilo koji dio koji je prethodno objavljen ili napisan
od strane drugih osoba, niti materijal koji je prihva\cj en u svrhu dodjele diplome bilo kojeg
stepena na bilo kojoj visoko\sh kolskoj ili nau\ch noj instituciji na svijetu. Ipak, obzirom na ozbiljnost
izrade jednog nau\ch nog rada, autor zadr\zh ava pravo da radi komparacije koristi i izvore
koji nisu njegovo originalno djelo, ali su obavezno navedeni u tekstu i referencama rada.

Autorska prava diplomskog rada ostaju pri autoru rada i Elektrotehni\ch kom fakultetu Univerziteta
u Sarajevu.

Ova izjava je napisana u skladu sa Pravilnikom o strukturi i sadr\zh aju doktorske disertacije i 
magistarskog rada na Elektrotehni\ch kom fakultetu u Sarajevu (br. 04-1-673/11, dana 17.01.2011.
godine).
\vfill
\begin{center}
\noindent\rule{8cm}{0.4pt}\\
Muhamed Pari\cj\ 
\vfill
U Sarajevu,\\
septembar 2017.
\end{center}

\tableofcontents

\chapter{Uvod}
Sada \cj emo se po\ch eti baviti slo\zh enijim tehnikama interpolacije koje mogu davati mnogo bolje rezultate od linearne interpolacije. U idealnom slu\ch aju, rezultiraju\cj i frejmovi \cj e izgledati potpuno prirodno.
Postoji vi\sh e klasa algoritama koji su u praksi kori\sh teni, od kojih \cj emo mi u ovom poglavlju objasniti tri: Tehnike zasnovane na faznoj korelaciji, upore\dj ivanju blokova, i rje\sh avanju diferencijalnih jedna\ch ina. % Dodati jos diskretne ako uspijem
Zajedni\ch ko svim ovim algoritmima jeste da njihov izlaz ne\cj e biti novi interpolirani frejm, ve\cj\ \textit{polje opti\ch kog toka}.
Cilj ovog poglavlja jeste da odgovori na sljede\cj a pitanja:
\begin{enumerate}
\item \SH ta su vektori pomaka?
\item \SH ta je opti\ch ki tok?
\item Kako mo\zh emo koristiti poznavanje opti\ch kog toga za kreiranje interpoliranih frejmova?
\item Kojim metodama mo\zh emo izra\ch unati opti\ch ki tok?
\end{enumerate}

Neka su zadana dva susjedna frejma, $A$ i $B$, izme\dj u kojih \zh elimo interpolirati novi frejm. Intuitivno gledaju\cj i, neki objekti frejma $A$ \cj e se tako\dj er nalaziti u frejmu $B$, samo na nekoj drugoj poziciji, blizu
svoje originalne. Drugim rije\ch ima, postoje vektori pomaka koji odgovaraju tim objektima, koji imaju dvije komponente i du\zh kojih su objekti prividno pomjereni izme\dj u frejmova $A$ i $B$.

Odre\dj ivanje vektora pomaka za objekte bi zahtjevalo tra\zh enje objekata u frejmovima, \sh to je izuzetno zahtjevno samo po sebi. Umjesto toga, svi algoritmi koje \cj emo objasniti \cj e pridru\zh ivati pojedina\ch ne
vektore pomaka svakom pikselu jednog od frejmova ($A$ ili $B$, neki algoritmi mogu odrediti koji je od ta dva slu\ch aja korisnije razmatrati). Algoritam \cj e jednostavno dobiti dva frejma, i za svaki piksel dati jedan
vektor, njegov vektor pomaka. Skup svih vektora pomaka piksela jednog frejma na drugi nazivamo poljem opti\ch kog toka, pri \ch emu opti\ch ki tok mo\zh emo definisati kao prividno kretanje objekata u video
sekvenci.

Najve\cj i dio posla pri generisanju interpoliranog frejma upravo jeste tra\zh enje kvalitetnog opti\ch kog toka. Nakon toga, proces je relativno jednostavan:
\begin{enumerate}
\item Svaki piksel pomjeriti du\zh\ svog vektora pomaka pomno\zh enog sa $\alpha$ (realan broj dobiven na isti na\ch in kao $\alpha$ u dijelu u linearnoj interpolaciji) te ga spasiti na tu poziciju.
\item Polja za koja nije prona\dj en niti jedan piksel popuniti kori\sh tenjem linearne interpolacije.
\end{enumerate}

\subsection{Kernel konvolucija} % http://web.pdx.edu/~jduh/courses/Archive/geog481w07/Students/Ludwig_ImageConvolution.pdf
Kernel konvolucija je tehnika \ch esto kori\sh tena za obradu slike, i ima veliki broj razli\ch itih primjena. Neke od njih uklju\ch uju detekciju ivica, izo\sh travanje, zamu\cj ivanje i druge razne filtere koji mogu biti primijenjeni
na slike. Ulazi kernel konvolucije su slika i sam kernel, dok je izlaz obra\dj ena slika. Kerneli funkcioni\sh u na samo jednom dvodimenzionalnom signalu, tako da \cj emo mi posmatrati samo intenzitete pojedinih piksela, 
ne njihove odvojene komponente (crvena, zelena i plava).

Kernel nije ni\sh ta drugo nego matrica realnih brojeva. Za na\sh e potrebe, dimenzije matrice \cj e morati biti neparni brojeva (vidjet \cj emo za\sh to je lak\sh e raditi sa takvim kernelima). Neka je zadana matrica intenziteta
piksela $A$, kernel $K$ i izlazna matrica $B$. Dimenzije izlazne matrice su iste kao dimenzije ulazne. Neka su dimenzije kernela (visina i \sh irina) $p$ i $q$. Svaki piksel izlazne matrice$B_{i,j}$ ra\ch unamo narednom formulom:

\[
B_{i,j}=\sum_{y=-p'}^{p'}\sum_{x=-q'}^{q'}A_{i+y,j+x}*K_{y+p'+1,x+q'+1} % POJEDNOSTATVITI FORMULU
\]
pri \ch emu su:
\[
p'=(p-1)/2
\]
\[
q'=(q-1)/2
\]
Zamislimo da smo postavili kernel iznad piksela sa koordinatama $i,j$, tako da element u sredini kernela ima upravo koordinate $i,j$. Ovdje vidimo za\sh to smo postavili ograni\ch enje da dimenzije kernela moraju biti neparni brojevi.
Zatim \cj emo pomno\zh iti svaki element kernela sa elementom matrice $A$ koji se nalazi ta\ch no ispod njega. Zbir svih tih proizvoda upisujemo u element $B_{i,j}$.

Kori\sh tenjem razli\ch itih kernela mo\zh emo dobiti razli\ch ite efekte. Ispod slijede primjeri nekih od \ch e\sh \cj e kori\sh tenih:
\vspace{10px} % Uljepsati sve
\[
\text{Detekcija ivica: } 
\begin{bmatrix}
-1 & -1 & -1 \\
-1 &  8 & -1 \\
-1 & -1 & -1
\end{bmatrix}
\]
\vspace{5px}
\[
\text{Zamu\cj ivanje: } 
\begin{bmatrix}
\frac{1}{9} & \frac{1}{9} & \frac{1}{9} \\
\frac{1}{9} &  \frac{1}{9} &\frac{1}{9} \\
\frac{1}{9} & \frac{1}{9} & \frac{1}{9}
\end{bmatrix}
\]
\vspace{5px}
\[
\text{Izo\sh travanje: } 
\begin{bmatrix}
0  & -1 &  0 \\
-1 &  5 & -1 \\
 0 & -1 &  0
\end{bmatrix}
\]

Neki od algoritama koje \cj emo objasniti koriste kernel konvoluciju, u opisu algoritama \cj emo navesti i konkretne kori\sh tene kernele.

\chapter{Osnovne tehnike}
\input{"chapters/Osnovne tehnike"}

\chapter{Opti\ch ki tok}
\input{"chapters/Opticki tok"}

\chapter{Uklanjanje gre\sh aka}
\input{"chapters/Uklanjanje gresaka"}

\chapter{Implementacija interpolatora kori\sh tenjem OpenCV biblioteke}
\input{"chapters/Implementacija interpolatora"}

\chapter{Zaklju\ch ak}
Interpolacija frejmova ima mnogobrojne realne primjene, od pobolj\sh anja kvalitete video zapisa, do video kompresije. Algoritmi i metode kori\sh teni za interpolaciju
tako\dj er imaju mnoge druge primjene. Opti\ch ki tok, prividno kretanje objekata kroz video zapis je od klju\ch ne va\zh nosti za mnoge oblasti ra\ch unarske vizije, poput
raznih vrsta robota, kamera koje detektuju, prate i mjere pokret, te prepoznavanja i pra\cj enja raznih oblika. U svrhu interpolacije smo objasnili nekoliko algoritama, od kojih
neki nisu uop\sh te zasnovani na izra\ch unavanju opti\ch kog toka, a drugi koriste slo\zh ene matemati\ch ke metode za precizno i efikasno ra\ch unanje. Neke metode,
poput obi\ch ne linearne interpolacije, su toliko brze da mogu biti kori\sh tene u realnom vremenu. Kao i u svim ostalim oblastima ra\ch unarskih nauka, jo\sh\ uvijek se radi
na tra\zh enju boljih, br\zh ih i efikasnijih metoda.

Od metoda koje izra\ch unaju opti\ch ki tok, najjednostavnije su zasnovane na \ch istom uparivanju blokova piksela i tra\zh enju blokova koji se najbolje podudaraju.
S obzirom da bi izra\ch unavanje kvalitete uparivanja svaka dva para blokova piksela bilo neprihvatljivo sporo, pretraga je ograni\ch ena na relativno mali broj potencijalnih
kandidata, i ubrzana heuristi\ch kim metodama koje se razlikuju od algoritma do algoritma.

Naprednije metode uklju\ch uju uparivanje ubrzano kori\sh tenjem brze Fourierove transformacije, rje\sh avanjem sistema linearnih jedna\ch ine, ili iterativnim pobolj\sh anjem
kvalitete opti\ch kog toka. Nakon izra\ch unavanja opti\ch kog toka, ideja je uvijek ista: piksele pomjerimo jedan dio puta du\zh\ njihovog vektora pomaka, ostale piksele izra\ch unamo
drugim metodama (poput linearne interpolacije), te popravimo koliko mo\zh emo nastale artefakte. Kvalitet generisanih frejmova nikada ne\cj e biti isti kao kvalitet originalnih,
ali kori\sh tenjem boljih i br\zh ih metoda, pribli\zh avamo se tom idealnom cilju.

\begin{thebibliography}{9}
\bibitem{blockmatching}
Barjatya, A, "Block Matching Algorithms for Motion Estimation"
\bibitem{lowcomplexity}
Zhai, J., Keman, Y., Li, J., Shipeng, L., "A Low Complexity Motion Compensated Frame Interpolation Method"
\bibitem{phaseonly}
Reyes, A., Alba, A., Arce-Santana, E. R., "Optical Flow Estimation using Phase Only-Correlation"
\end{thebibliography}
\end{document}