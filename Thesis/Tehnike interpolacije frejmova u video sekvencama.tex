\documentclass[12pt]{report}

\usepackage[utf8]{inputenc}
\usepackage{graphicx}
\usepackage[T1]{fontenc}
\usepackage{lipsum}
\usepackage[hidelinks]{hyperref}
\usepackage{textcomp}

\graphicspath{ {images/} }

%%%%%%%%%%%%%%%%%%%%%%%%%%%
\makeatletter
\def\@makechapterhead#1{%
  \vspace*{50\p@}%
  {\parindent \z@ \raggedright \normalfont
    \interlinepenalty\@M
    \Huge \bfseries #1\par\nobreak
    \vskip 40\p@
  }}
\def\@makeschapterhead#1{%
  \vspace*{50\p@}%
  {\parindent \z@ \raggedright
    \normalfont
    \interlinepenalty\@M
    \Huge  \bfseries  #1\par\nobreak
    \vskip 40\p@
  }}
\makeatother
%%%%%%%%%%%%%%%%%%%%%%%%%%%
% Kod iznad preuzet (uz manje modifikacije) sa: http://tex.stackexchange.com/a/120743
% Bez ovog koda iznad svakog poglavlja bi pisalo 'Chapter 1', 'Chapter 2', itd.

\newcommand{\ch}{\v{c}}
\newcommand{\CH}{\v{C}}
\newcommand{\cj}{\'{c}}
\newcommand{\CJ}{\'{C}}
\newcommand{\sh}{\v{s}}
\newcommand{\SH}{\v{S}}
\newcommand{\zh}{\v{z}}
\newcommand{\ZH}{\v{Z}}
% Radi brzeg pisanja afrikata (\dj i \DJ su vec definisani)

\newcommand{\midtilde}{\raisebox{0.6ex}{\texttildelow}} % Iz nekog razloga nije bas jednostavno napisati tildu u LaTex-u, ali ovo radi ok
\renewcommand{\contentsname}{Sadr\zh aj}

\begin{document}

\tableofcontents

\chapter{Uvod}
\section{Op\cj enito o video sekvencama}
Radi jednostvnosti, mi \cj emo imati relativno jednostavan pogled na video sekvence. Fokus rada je na direktnoj manipulaciji nad frejmovima (sli\ch icama) i pikselima radi dobivanja krajnje sekvence, 
ne u algoritmima dekodiranja i enkodiranja video fajlova, koji mogu biti veoma slo\zh eni. Konkretno, mi \cj emo na video sekvencu gledati kao obi\ch an niz frejmova. Svaki frejm je matrica piksela. 
Veli\ch ina frejma naravno zavisi od veli\ch ine konkretnog fajla na kojem radimo, ali za primjere \cj emo koristiti frejmove dimenzija 1280x720 piksela (1280 piksela \sh irine i 720 piksela visine).
 Video sekvenc  ove dimenzije se \ch esto zovu HD (\textit{High Definition}) video sekvence. Neke druge \ch esto kori\sh tene dimenzije uklju\ch uju 640x360, 1920x1080 (\textit{Full HD}), 2560x1440 
 (\textit{2K}), 3840x2160(\textit{4K}), itd. HD veli\ch ina je tako\dj er pogodno jer jedan frejm sadr\zh i \midtilde 1 milion piksela (ta\ch no 921600), \sh to prora\ch une \ch ija preciznost nije od izrazite 
 va\zh nosti \ch ini lak\sh im. 
 
 Jo\sh\ jedna va\zh na osobina video sekvenci jeste \textit{framerate}, broj frejmova u sekundi. Framerate video sekvence zavisi od primjene. Za filmove, standardni framerate
iznosi 24 \textit{fps} (\textit{frames per second}, broj frejmova u sekundi). Video igre naj\ch e\sh \cj e prikazuju 30 ili 60 fps. Ve\cj ina monitora ne mo\zh e prikazivati vi\sh e od 60 fps. Iako predstavlja
va\zh nu osobinu video sekvenci, nama framerate nije od naro\ch ite va\zh nosti. Ulaz i izlaz predstavljaju sekvence frejmova, koje mogu biti prikazane proizvoljno \ch esto. O tome koji framerate
je pogodan za koju namjenu \cj e biti vi\sh e govora u dijelu o primjenama video interpolacije

Zavisno od toga u kojoj smo fazi obrade frejma, piksel predstavlja jednu od dvije razli\ch ite stvari. Po\ch etni i krajnji video fajlovi imaju piksele sa 3 komponente: Crvena, zelena i plava. Svaka 
komponenta ima cjelobrojnu ja\ch inu izme\dj u 0 i 255, uklju\ch ivo. Me\dj utim, u fazi obrade, \ch esto je pogodnije gledati na svaki piksel kao samo jedan broj (u istom opsegu) koji
predstavlja ja\ch inu svjetlosti piksela (drugim rije\ch ima, frejm postaje crno-bijel). Intenzitet svjetlosti svakog piksela \cj e biti obi\ch na aritmeti\ch ka sredina crvene, zelene i plave komponente.

\subsection{Video fajlovi}
Iako fokus rada nije na video fajlovima, vrijedi spomenuti nekoliko detalja. Proces \ch itanja i pisanja video sekvenci u fajlove \cj e obavljati OpenCV biblioteka (o kojoj \cj e biti vi\sh e rije\ch i u 
posljednjem poglavlju). \CH uvati video fajlove kao obi\ch ne nizove frejmova bi bilo izuzetno neefikasno. Naime, video HD dimenzije du\zh ine 2 sata (pri \ch emu svaki piksel zauzima 3 bajta i
framerate iznosi 24 fps) bez ikakve kompresije bi zauzimao $2 * 3600 * 24 * 1280 * 720 * 3 = 477757440000$ bajta (\midtilde 466 GB), gdje u stvarnosti ta veli\ch ina iznosi naj\ch e\sh \cj e 
nekoliko gigabajta. Kompresija video fajlova se zasniva na 2 osnovna principa: \textit{intraframe} kompresija i \textit{interframe} kompresija. Intraframe kompresija se bavi smanjivanjem veli\ch ine
svakog pojedina\ch nog frejma, \sh to nema dodirnih ta\ch aka sa tehnikama interpolacije frejmova. Interframe kompresija poku\sh ava smanjiti veli\ch inu krajnjeg fajla tako \sh to tra\zh i sli\ch nosti
izme\dj u uzastopnih frejmova, te umjesto spa\sh avanja \ch itavih frejmova, spasi razlike izme\dj u trenutnog i pro\sh log frejma (ili trenutnog i narednog). Poglavlje 3 je posve\cj eno upravo tehnikama
koje tra\zh e sli\ch nosti izme\dj u frejmova. Osim video sadr\zh aja, video fajlovi sadr\zh e i druge komponente poput jedne ili vi\sh e audio komponente, prijevoda, menija, itd. Ove komponente
ne\cj e biti obra\dj ivane u ovom radu, niti ih OpenCV biblioteka podr\zh ava.

\section{Primjene video interpolacije}

\chapter{Osnovne tehnike}
\input{"chapters/Osnovne tehnike"}

\chapter{Algoritmi uparivanja blokova}
\input{"chapters/Algoritmi uparivanja blokova"}
%Klasicni i sa kros-korelacijom

\chapter{Uklanjanje gre\sh aka}
\input{"chapters/Uklanjanje gresaka"}
%Popravljanje vektorskog polja pomaka, popunjavanje rupa, ivice, OBMC

\chapter{Interpolacija frejmova kori\sh tenjem opti\ch kog toka}
%input...
%Detekcija uglova, LK, Farneback

%\chapter{Paralelno izvr\sh avanje algoritama interpolacije}
%input...
%Pricati o mogucnosti paralelizacije svih spomenutih algoritama

\chapter{Implementacija interpolatora kori\sh tenjem OpenCV biblioteke i benchmark testovi}
%input...
%Pricati o svemu spomenutom, samo kroz OpenCV

\chapter{Zaklju\ch ak}
%input...

\end{document}