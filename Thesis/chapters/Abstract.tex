This work explains some key terms related to video sequence interpolation, such as frames, motion vectors, and optical flow. After that, we explain how one might
interpolate frames if the optical flow is known, the problems that arise from that approach, and how to solve them. The main part of the work deals with optical flow
algorithms, categorized into block matching, phase correlation, or differential equation based methods. Basic usage of these algorithms generates frames
of poor quality, hence we explain some methods used to correct errors and artifacts that arise during the process. The final chapter explains the implementation
of an interpolator program, written in the C++ programming language, and using the OpenCV library.