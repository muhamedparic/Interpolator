Algoritmi opisani u ovom radu se mogu relativno jednostavno implementirati u bilo kojem modernom programskom jeziku. Zapravo, najve\cj i problem predstavlja \ch itanje i pisanje video datoteka. Naime,
za to nam je potreban na\ch in da pozovemo odgovaraju\cj i \textit{codec} (eng. \textit{coder-decoder}) za format video datoteke koji koristimo. OpenCV biblioteka nam omogu\cj ava upravo \ch itanje
i pisanje video datoteka, te direktan pristup pikselima svakog frejma, \sh to je nama od klju\ch nog zna\ch aja za svaki opisani algoritam.

\subsection{OpenCV} %http://opencv.org/about.html
OpenCV (eng. \textit{Open Computer Vision} je biblioteka originalno razvijena od strane Intel-a, koja se fokusira na oblasti ra\ch unarske vizije i ma\sh inskog u\ch enja. Originalna implementacija biblioteke
je u C++ programskom jeziku, dok tako\dj er postoji podr\sh ka za C, Java, Python i MATLAB. Funkcionalnosti biblioteke koje su nama potrebne uklju\ch uju \ch itanje i pisanje video datoteka sa punom
kontrolom nad atributima (poput formata, rezolucije, framerate-a, itd.), direktna kontrola nad pikselima svakog pojedina\ch nog frejma, pronala\zh enje opti\ch kog toka kroz ugra\dj ene funkcije, te neke
druge funkcionalnosti poput mogu\cj nosti prikazivanja bilo kojeg frejma na ekranu u bilo koje vrijeme. Sve ovo predstavlja jedan veoma mali dio mogu\cj nosti biblioteke, koja ima preko 2500 implementiranih
i optimizovanih algoritama., zbog \ch ega je kori\sh tena u mnogim od najve\cj ih tehnolo\sh kih kompanija na svijetu.

Tri najva\zh nije OpenCV klase koje \cj emo koristiti za implementaciju su \lstinline{cv::Mat}, \lstinline{cv::VideoCapture}, te \lstinline{cv::VideoWriter}. Sve klase i funkcije OpenCV biblioteke se nalaze u \lstinline{cv}
imenskom prostoru. \lstinline{cv::Mat} je vi\sh enamjenska klasa koju OpenCV koristi za dr\zh anje raznih tipova matrica, pri \ch emu svaki element mo\zh e imati jedan ili vi\sh e razli\ch itih kanala. O\ch igledna
korist ovoga jeste da mo\zh emo istu klasu korsititi za dr\zh anje frejmova u boji (gdje svaki piksel ima 3 kanala) i samo intenziteta svakog piksela. Na primjer, ako imamo varijablu \lstinline{frejm} koja je tipa
\lstinline{cv::Mat}, da bismo pristupili vrijednosti piksela u boji na koordinatama $(x,y)$, koristimo sljede\cj i kod:
\begin{lstlisting}
cv::Vec3b piksel = frejm.at<cv::Vec3b>(y, x);
\end{lstlisting}

Pri tome, \lstinline{cv::Vec3b} je klasa koja dr\zh i tri bajta (jedan za svaki kanal), kojima mo\zh emo pristupiti kori\sh tenjem subscript operatora \lstinline{piksel[0]}, \lstinline{piksel[1]} i \lstinline{piksel[2]}. 
I \lstinline{.at} metoda i subscript operator vra\cj aju reference, tako da je isti pristup mogu\cj e koristiti i za postavljanje vrijednosti piksela, bilo \ch itavih ili pojedina\ch nih komponenti. 
Jo\sh\ treba napomenuti da subscript operatori ne referenciraju kanale piksela uobi\ch ajnim RGB (crvena-zelena-plava) redoslijedom, nego obrnutim (BGR). \lstinline{cv::Mat} tako\dj er ima \lstinline{.rows} i 
\lstinline{.cols} atribute, koji sadr\zh e broj redova odnosno broj kolona frejma (drugim rije\ch ima, rezoluciju).

\lstinline{cv::VideoCapture} i \lstinline{cv::VideoWriter} klase koristimo za \ch itanje odnosno pisanje video datoteka. Pri tome, \lstinline{cv::VideoCapture} mo\zh e uzimati video podatke i iz drugih izvora,
poput web kamera. U oba slu\ch aja, \lstinline{.open} metoda se koristi za otvaranje datoteke za \ch itanje odnosno pisanje (pri \ch emu u slu\ch aju pisanja tako\dj er navodimo potrebne informacije poput
formata, rezolucije, i framerate-a video zapisa), tok se \lstinline{.release} koristi za pravilno zatvaranje. \lstinline{cv::VideoCapture} implementira \lstinline{.read} metodu, ili ekvivalentni \lstinline{>>} operator koji
pro\ch ita, dekodira, i vrati naredni nepro\ch itani frejm (vrijednost tipa \lstinline{cv::Mat}. Da bismo znali da smo do\sh li do kraja video zapisa, preporu\ch eni na\ch in jeste poziv metode \lstinline{.empty()} 
nad vra\cj enim frejmom, koja \cj e vratiti logi\ch ku vrijednost \lstinline{true} ako smo ve\cj\ pro\ch itali posljednji frejm video zapisa, zbog \ch ega \cj e svaki drugi poku\sh aj \ch itanja rezultirati praznim frejmom.

Prate\cj i istu logiku, \lstinline{cv::VideoWriter} implementira \lstinline{.write} metodu, umjesto koje mo\zh emo koristiti \lstinline{<<} operator. Parametar u oba slu\ch aja predstavlja vrijednost tipa
\lstinline{cv::Mat}, koja \cj s biti dodana na kraj video zapisa. Pozivom prije spomenute metode, video zapis \cj e biti zapisan u datoteku. Ako parametri proslije\dj eni konstruktoru \lstinline{cv::VideoWriter}
objekta ne odgovaraju formatu frejmova koje poku\sh avamo dodati u video zapis, napravljena datoteka \cj e biti veli\ch ine svega nekoliko kilobajta i ne\cj e sadr\zh avati nikakve korisne informacije.

% Dodati Farneback-a, ako bude koristen

Osnovni tok interpolatora je sljede\cj ii:
\begin{enumerate}
\item Inicijaliziraj objekte za \ch itanje i pisanje video datoteka
\item U\ch itaj prethodni frejm
\item U\ch itaj naredni frejm
\item Inicijaliziraj objekat za pronala\zh enje opti\ch kog toka, te pozovi metodu za izvr\sh avanje
\item Prona\dj eni opti\ch ki tok proslijedi metodi za generaciju
\end{enumerate}