Cilj ovog rada jeste da objasni osnovne pojmove vezane za video sekvence, te samo neke od brojnih i raznovrsnih algoritama koji se u praksi koriste za video interpolaciju. Objasnit \cj emo \sh ta je to ta\ch no video sekvenca, opti\ch ki tok,
interpolacija, framerate, i jo\sh\ mnogo toga. Algoritmi video interpolacije su veoma specijalizovani, te imaju relativno usko podr\ch je primjene. Me\dj utim, ipak su izuzetno korisni, te \cj emo se sa konkretnim podru\ch jima primjene uskoro
upoznati. Prije toga bi ipak bilo korisno napraviti kratak uvod u video sekvence, te u to kako ih ra\ch unari vide.

\section{Video sekvence}
Ljudi svijet oko sebe vide kao sliku koja se kontinuirano mijenja. To jeste, svaki vremenski trenutak je razli\ch it od pro\sh log, koliko god da je blizu. Ljudi i objekti se kre\cj u "glatko", bez potpuno naglih pokreta. Tako\dj er, slike koje vidimo
su prostorno kontinualne, prividno beskona\ch no detaljne. Ra\ch unari, s druge strane, moraju slike i video fajlove \ch uvati u ograni\ch enoj koli\ch ini memorije, \sh to zna\ch i da mora do\cj i do kompromisa. Ve\cj a koli\ch ina detalja
zahtijeva vi\sh e memorije za \ch uvanje. Svaka slika ima samo kona\ch nu koli\ch inu detalja, \sh to zna\ch i da se sastoji od kona\ch nog broja piksela. Piksel (od eng. \textit{picture element}) je najmanji djeli\cj\ slike. Dva naj\ch e\sh \cj a tipa piksela,
pri \ch emu \cj emo mi u ovom radu koristiti oba, su 1-komponentni i 3-komponentni (nekada se umjesto termina komponenta koristi termin kanal). Jednokomponentni piksel je  djeli\cj\ slike koji je jednobojan. 1-komponentni pikseli su
gotovo islju\ch ivo monohromatski, tj. njihova boja mo\zh e biti crna, bijela, ili neka nijansa sive. 3-komponentni pikseli, s druge strane, se sastoje od 3 jednobojne komponente, koje zovemo podpikseli ili subpikseli. Ako je piksel oblika kvadrata,
onda su podpikseli oblika pravougaonika (koji su po pravilu izdu\zh eni vertikalno). Na svim modernim ekranima, boje podpiksela su crvena, zelena i plava. Ovaj raspored \cj emo ubudu\cj e zvati RGB (od eng. \textit{Red-Green-Blue}). Drugi tipovi podpiksela
 tako\dj er postoje, koji se moraju konvertovati u RGB za prikaz na ekranima.
Svaki ekran i svaka slika se sastoje od odre\dj enog broja piksela, koji su po pravilu raspore\dj eni u matricu. Ako na monitoru
gledamo sliku koja ima vi\sh e horizontalnih ili vertikalnih piksela od na\sh eg monitora, onda ne mo\zh emo vidjeti sve detalje te slike. Koji detalji \cj e biti izostavljeni zavisi od konkretnog algoritma kori\sh tenog za skaliranje slike, o kojima
ne\cj emo pri\ch ati u ovom radu.

Objasnili smo kako ra\ch unari rje\sh avaju problem prostorno kontinualnih slika, sada \cj emo objasniti video sekvence. Isto kao \sh to smo sliku rastavili u piksele, video \cj emo rastaviti u \textit{frejmove}. Frejmovi su pojedina\ch ne sli\ch ice (ili okvirovi)
od kojih se sastoji video sekvenca (sekvenca upravo zato \sh to je to sekvenca frejmova). Svaki frejm provede odre\dj eno vrijeme na ekranu, nakon \ch ega se mijenja narednim frejmom. U pravilu, to vrijeme je za svaki frejm isto (ako je to mogu\cj e), da
bi video sekvenca koju gledamo izgledala \sh to prirodnije. Broj frejmova koje vidimo svake sekunde (odnosno broj recipro\ch an broju sekundi koji svaki frejm provede na ekranu) se zove \textit{framerate}, a kori\sh tena jedinica je \textit{frame per second}, 
skra\cj eno \textit{fps}. Tako je maksimalni framerate ve\cj ine ekrana 60 fps, framerate filmova je gotovo uvijek 24 fps, framerate igrica zavisi od mnogo faktora, ali je u ve\cj ini slu\ch ajeva 30 ili 60 fps.

\section{Interpolacija u video sekvenci}
Sada kada znamo \sh ta je framerate, mo\zh emo objasniti video interpolaciju. Naime, interpolacijom video sekvence se dobivaju novi frejmovi izme\dj u postoje\cj ih. Taj novi frejm \cj e biti veoma sli\ch an svojim susjednim, jer u su\sh tini predstavlja
"me\dj u korak". Izlaz algoritma za video interpolaciju \cj e biti video veoma sli\ch an po\ch etnom, sa dodatnim frejmovima koji su generisani iz originalnih. U matemati\ch kom smislu, interpolacija predstavlja izra\ch unavanje novih vrijednosti na osnovu
okolnih poznatih, \sh to zvu\ch i veoma sli\ch no video interpolaciji. Me\dj utim, zbog svoje prirode, algoritmi interpolacije video sekvenci se potpuno razlikuju od onih kori\sh tenih u matematici.

\section{Primjene video interpolacije}
Kao \sh to smo ve\cj\ rekli, izlaz algoritma za video interpolaciju jeste novi video koji je veoma sli\ch an originalnom. Ako \zh elimo, mo\zh emo koristiti taj video, koji \cj e zbog svog vi\sh eg framerate-a izgledati prirodnije (pod uslovnom da je kvalitetno
interpoliran) i gla\dj e. Tako bi sportski sadr\zh aj mogao biti sniman kamerom visoke rezolucije, ali relativno niskog fremerate-a, nakon \ch ega bi frejmovi mogli biti interpolirani da bismo dobili video koji je i visoke kvalitete i visokog framerate-a.
Tako\dj er bismo mogli koristiti video interpolaciju za video pozive, gdje je mre\zh na propusnost \ch esto ograni\ch avaju\cj i faktor. Mo\zh emo snimiti video niskog framerate-a, koji ne zahtijeva visoku mre\zh nu propusnost, te interpolirati
dolaze\cj i video. Naravno, u oba ova slu\ch aja \cj e dobiveni rezultat biti ni\zh eg kvaliteta nego video sniman visokim framerate-om, ali uz kvalitetne algoritme razlika ne\cj e biti zna\ch ajna.

Postoje ve\cj\ projekti koji koriste video interpolaciju za kreiranje \textit{slow-motion} video sekvenci. Umjesto zadr\zh avanje ukupne du\zh ine video sekvencei pove\cj avanja framerate-a, mo\zh emo pove\cj ati du\zh inu i zadr\zh ati framerate.
Jedan primjer bi bio \href{https://github.com/dthpham/butterflow}{\textcolor{blue}{Butterflow}}.