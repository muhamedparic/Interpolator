Interpolacija frejmova ima mnogobrojne realne primjene, od pobolj\sh anja kvalitete video zapisa, do video kompresije. Algoritmi i metode kori\sh teni za interpolaciju
tako\dj er imaju mnoge druge primjene. Opti\ch ki tok, prividno kretanje objekata kroz video zapis je od klju\ch ne va\zh nosti za mnoge oblasti ra\ch unarske vizije, poput
raznih vrsta robota, kamera koje detektuju, prate i mjere pokret, te prepoznavanja i pra\cj enja raznih oblika. U svrhu interpolacije smo objasnili nekoliko algoritama, od kojih
neki nisu uop\sh te zasnovani na izra\ch unavanju opti\ch kog toka, a drugi koriste slo\zh ene matemati\ch ke metode za precizno i efikasno ra\ch unanje. Neke metode,
poput obi\ch ne linearne interpolacije, su toliko brze da mogu biti kori\sh tene u realnom vremenu. Kao i u svim ostalim oblastima ra\ch unarskih nauka, jo\sh\ uvijek se radi
na tra\zh enju boljih, br\zh ih i efikasnijih metoda.

Od metoda koje izra\ch unaju opti\ch ki tok, najjednostavnije su zasnovane na \ch istom uparivanju blokova piksela i tra\zh enju blokova koji se najbolje podudaraju.
S obzirom da bi izra\ch unavanje kvalitete uparivanja svaka dva para blokova piksela bilo neprihvatljivo sporo, pretraga je ograni\ch ena na relativno mali broj potencijalnih
kandidata, i ubrzana heuristi\ch kim metodama koje se razlikuju od algoritma do algoritma.

Naprednije metode uklju\ch uju uparivanje ubrzano kori\sh tenjem brze Fourierove transformacije, rje\sh avanjem sistema linearnih jedna\ch ine, ili iterativnim pobolj\sh anjem
kvalitete opti\ch kog toka. Nakon izra\ch unavanja opti\ch kog toka, ideja je uvijek ista: piksele pomjerimo jedan dio puta du\zh\ njihovog vektora pomaka, ostale piksele izra\ch unamo
drugim metodama (poput linearne interpolacije), te popravimo koliko mo\zh emo nastale artefakte. Kvalitet generisanih frejmova nikada ne\cj e biti isti kao kvalitet originalnih,
ali kori\sh tenjem boljih i br\zh ih metoda, pribli\zh avamo se tom idealnom cilju.