Direktnom primjenom algoritama uparivanja blokova, mogu\cj e je dobiti interpolirane frejmove. Me\dj utim, ti frejmovi \cj e biti veoma slabe kvalitete. U ovom poglavlju \cj e biti razra\dj ene tehnike pomo\cj u kojih
\cj emo rije\sh iti sljede\cj a 3 problema:
\begin{enumerate}
	\item Postojat \cj e velike "rupe" u interpoliranom frejmu, jer za zna\ch ajan dio piksela ne\cj e biti prona\dj en niti jedan blok koji ih pokriva. U prvom dijelu \cj e biti obja\sh njena tehnika koju \cj emo koristiti
	za njihovo popravljanje.
	\item Bez obzira na to koji algoritam uparivanja koristimo, za neke blokove \cj e biti generisani neispravni vektori pomaka. U drugom dijelu \cj e biti obja\sh njene neke tehnike za detekciju i korekciju
	neispravnih vektora pomaka.
	\item U zavisnosti od toga koji algoritam uparivanja koristimo, postoji ni\zh a ili vi\sh a vjerovatno\cj a da \cj e granice izme\dj u blokova u interpoliranom frejmu biti veoma o\sh tre i o\ch igledne. Ova 
	\ch injenica opet proizlazi iz nesavr\sh enosti algoritama uparivanja. Da bismo u\ch inili interpolirane frejmove realisti\ch nijim i ugodnijim za gledati, u tre\cj em dijelu \cj e biti razra\dj ene neke tehnike pomo\cj u kojih je
	mogu\cj e umanjiti ovaj efekat.
\end{enumerate}

\section{Popunjavanje rupa}

\section{Neispravni vektori pomaka}
Kori\sh tenjem algoritama uparivanja blokova, dobili smo odgovaraju\cj e vektore pomaka za svaki blok veli\ch ine 16x16 piksela. Me\dj utim, postoji visoka vjerovatno\cj a da neki od tih vektora ne odgovaraju stvarnom pokretu
u video sekvenci. Prvo \cj emo definisati \sh ta zapravo zna\ch i neispravan vektor pomaka, kako ih detektovati, a zatim kako ih popraviti.

\subsection{Detekcija neispravnih vektora pomaka}

\subsection{Ispravljanje neispravnih vektora pomaka}

\section{Izgla\dj ivanje granica izme\dj u blokova}