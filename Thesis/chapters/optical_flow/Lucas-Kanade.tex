Osnovna ideja iza Lucas-Kanade metode jeste kori\sh tenje vi\sh e od jednog piksela. Kori\sh tenjem okoline piksela \ch iji opti\ch ki tok poku\sh avamo izra\ch unati, dobit \cj emo vi\sh e jedna\ch ina
(npr. ako koristimo 3x3 podru\ch je oko zadanog piksela dobit \cj emo 9 nezavisnih jedna\ch ina). Naravno, sistem od 9 jedna\ch ina sa 2 nepoznate gotovo nikada nema jedinstveno rje\sh enje,
tako da Lucas-Kanade metoda pronalazi najbolje rje\sh enje metodom najmanjih kvadrata. Nije nu\zh no koristiti ta\ch no 9 jedna\ch ina, metoda radi sa bilo kojim brojem piksela u okolini zadanog.

Umjesto pisanja jedna\ch ine opti\ch kog toka 9 puta za svih 9 piksela, mo\zh emo svih 9 predstaviti jednostavnom matri\ch nom jedna\ch inom\cite{matlablk, lknutshell}:
\[
S
\begin{pmatrix}
u \\
v
\end{pmatrix}
=\vec{t}
\]

pri \ch emu je $S$ 9x2 matrica \ch iji su redovi vektori oblika 

\[
\begin{pmatrix}I_x(x+p,y+q),I_y(x+p,y+q)\end{pmatrix},p,q\in (-1,0,1)
\] 

dok je $\vec{t}$ vektor-kolona \ch iji su \ch lanovi
$-I_t(x+p,y+q),p,q\in (-1,0,1)$. Cilj nam je prona\cj i $u$ i $v$ takve da je zbir kvadrata razlika lijevih i desnih strana jedna\ch ina minimalan.

Rje\sh enje dobijamo tako \sh to pomno\zh imo slijeva obje strane sa $S^T$:
\[
S^TS\begin{pmatrix}u \\ v\end{pmatrix}=S^T\vec{t}
\]
a zatim pomno\zh imo obje strane slijeva sa $(S^TS)^{-1}$:
\[
\begin{pmatrix}u \\ v\end{pmatrix}=(S^TS)^{-1}S^T\vec{t}
\]

Izra\ch unati elemente matrice $I_t$ nije te\sh ko, jer mo\zh emo samo uzeti razliku vrijednosti piksela narednog i trenutnog frejma, odnosno defini\sh emo:
\[
I_t(x,y,t)=I(x,y,t+1)-I(x,y,t)
\]

Prostorne izvode $I_x,I_y$ bismo mogli izra\ch unati najjednostavnije formulama:
\[
I_x(x,y,t)=I(x+1,y,t)-I(x,y,t)
\]
\[
I_y(x,y,t)=I(x,y+1,t)-I(x,y,t)
\]

Ovi izvodi odgovaraju kori\sh tenju kernela $\begin{bmatrix}0 & -1 & 1\end{bmatrix}$, odnosno njegove transponovane forme. Tako\dj er mo\zh emo koristiti
slo\zh enije kernele, u nadi da \cj emo tako dobiti vi\sh e informacija o prostornim parcijalnim izvodima. Jedan primjer bi mogao biti kernel:
\[
\begin{bmatrix}
\frac{1}{12} & -\frac{8}{12} & 0 & \frac{8}{12} & -\frac{1}{12}
\end{bmatrix}
\]

za $I_x$, te njegova transponovana forma za $I_y$.

S obzirom da sada imamo matrice $I_x,I_y,I_t$, mo\zh emo jednostavno izra\ch unati matricu $S$ i vektor $\vec{t}$. Prije toga mo\zh emo izgladiti matrice $I_x$, $I_y$ i $I_t$ kori\sh tenjem kernela:
\[
\frac{1}{256}
\begin{bmatrix}
1 & 4 & 6 & 4 & 1 \\
4 & 16 & 24 & 16 & 4 \\
6 & 24 & 36 & 24 & 6 \\
4 & 16 & 24 & 16 & 4 \\
1 & 4 & 6 & 4 & 1
\end{bmatrix}
\]

Ostala nam je jo\sh\ samo jedna prepreka, a to je invertovanje matrice $S^TS$. Bolje re\ch eno, s obzirom da se radi o 2x2 matrici, samo invertovanje je mogu\cj e uraditi jednostavnom formulom:
\[
\begin{bmatrix}
a & b \\
c & d
\end{bmatrix}^{-1}
=\frac{1}{ad-bc}
\begin{bmatrix}
d & -b \\
-c & a
\end{bmatrix}
\]

Prije nego \sh to mo\zh emo to uraditi, moramo provjeriti da li je matrica singularna, \sh to \cj emo uraditi tako \sh to prvo izra\ch unamo svojstvene vrijednosti matrice, te provjerimo da li su obje
dovoljno velike. Svojstvene vrijednosti $\lambda_1$ i $\lambda_2$ matrice dimenzija 2x2 mo\zh emo dobiti kao rje\sh enja jedna\ch ine:
\[
det
\begin{bmatrix}
a-\lambda_1 & b \\
c & d-\lambda_2
\end{bmatrix}
=0
\]

Ako su i $\lambda_1$ i $\lambda_2$ ispod nekog praga $\tau$, matrica je singularna, te je opti\ch ki tok nula-vektor. Ako je jedna od svojstvenih vrijednosti ve\cj a od $\tau$, matrica je lo\sh e uslovljena. 
Ako su obje svojstvene vrijednosti iznad praga $\tau$, matrica nije singularna, te mo\zh emo izra\ch unati opti\ch ki tok.

%Objasniti ovo gore

Sli\ch no kao standardni algoritmi uparivanja blokova, Lucas-Kanade metoda ima problema u slu\ch ajevima kada postoji veliki broj potencijalnih kandidata. Zbog toga se ova metoda naj\ch e\sh \cj e koristi
za pronala\zh enje \textit{rijetkog} opti\ch kog toka, odnosno toka koji je poznat za samo neke piksele. Ako \zh elimo koristiti ovu metodu za interpoliranje frejmova, moramo prona\cj i vektore pomaka za sve
piksele kori\sh tenjem podataka koje dobijemo iz malog broja ta\ch aka. 
Ako nas samo zanima rijetki opti\ch ki tok, trebamo na\cj i piksele \ch ije \cj e nam kretanje biti lako pratiti. U praksi se za tu svrhu koriste pikseli na uglovima, jer pomjeranje tih piksela i njihove okoline
\cj e uvijek biti lako uo\ch ljivo. Ako umjesto uglova koristimo npr. ivice, pomjeranje du\zh\ ivice nije uo\ch ljivo. Popularan i pouzdan algoritam za detekciju uglova jeste Shi-Tomasi algoritam\cite{opencvshitomasi}.