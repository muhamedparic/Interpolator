Osnovna ideja iza Lucas-Kanade metode jeste kori\sh tenje vi\sh e od jednog piksela. Kori\sh tenjem okoline piksela \ch iji opti\ch ki tok poku\sh avamo izra\ch unati, dobit \cj emo vi\sh e jedna\ch ina
(npr. ako koristimo 3x3 podru\ch je oko zadanog piksela dobit \cj emo 9 nezavisnih jedna\ch ina). Naravno, sistem od 9 jedna\ch ina sa 2 nepoznate gotovo nikada nema jedinstveno rje\sh enje,
tako da Lucas-Kanade metoda pronalazi najbolje rje\sh enje metodom najmanjih kvadrata. Nije nu\zh no koristiti ta\ch no 9 jedna\ch ina, metoda radi sa bilo kojim brojem piksela u okolini zadanog.

Umjesto pisanja jedna\ch ine opti\ch kog toka 9 puta za svih 9 piksela, mo\zh emo svih 9 predstaviti jednostavnom matri\ch nom jedna\ch inom:
\[
S
\begin{pmatrix}
u \\
v
\end{pmatrix}
=\vec{t}
\]

%pri \ch emu je $S$ 9x2 matrica \ch iji su redovi vektori oblika $\begin{pmatrix}I_x(x+p,y+q),I_y(x+p,y+q)\end{pmatrix},p,q\in (-1,0,1)$, a $\vec{t}$ vektor-kolona \ch iji su \ch lanovi
%$-I_t(x+p,y+q),p,q\in (-1,0,1)$. Cilj nam je prona\cj i $u$ i $v$ takve da je zbir kvadrata razlika lijevih i desnih strana jedna\ch ina minimalan.

Rje\sh enje dobijamo tako \sh to pomno\zh imo slijeva obje strane sa $S^T$:
\[
S^TS\begin{pmatrix}u \\ v\end{pmatrix}=S^T\vec{t}
\]
a zatim pomno\zh imo obje strane slijeva sa $(S^TS)^{-1}$:
\[
\begin{pmatrix}u \\ v\end{pmatrix}=(S^TS)^{-1}S^T\vec{t}
\]
\textit{Poglavlje nije dovr\sh eno}