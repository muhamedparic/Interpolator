%http://www.cs.utah.edu/~ssingla/IP/P4/Index.html
%http://www.sciencedirect.com/science/article/pii/S2212017313000145
Fazna korelacija je tehnika koju mo\zh emo koristiti za brzo pronala\zh enje mjesta preklapanja dviju sli\ch nih slika, \sh to nama upravo i treba. Mo\zh emo uzeti jedan blok piksela jednog frejma, te, koriste\cj i faznu korelaciju, brzo prona\cj i
gdje se taj blok piksela nalazi u drugom frejmu (ili barem neki drugi, veoma sli\ch an blok). Direktnom primjenom formule bismo dobili algoritam koji je zapravo veoma sli\ch an potpunoj pretrazi, koja je opisana u poglavlju o algoritmima
uparivanja blokova. Me\dj utim, mo\zh emo koristiti \textit{Brzu Fourierovu Transformaciju} (FFT) za zna\ch ajno ubrzanje. S obzirom na to da jedan korak algoritma uklju\ch uje izra\ch unavanje Hadamardovog proizvoda dvije matrice
(koji je jedino mogu\cj e izra\ch unati ako su matrice istih dimenzija), ovaj algoritam je ograni\ch en na tra\zh enje jednog ili vi\sh e mjesta poklapanja dva bloka piksela iste veli\ch ine. Tako\dj er, ako \zh elimo koristiti neki od FFT algoritama,
dimenzije blokova moraju biti stepeni broja dva. Mo\zh emo koristiti 16x16 blokove, dimenzije koje smo tako\dj er koristili u poglavlju u algoritmima uparivanja blokova. 

Koraci algoritma:
