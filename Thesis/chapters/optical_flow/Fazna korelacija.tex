%http://www.cs.utah.edu/~ssingla/IP/P4/Index.html
%http://www.sciencedirect.com/science/article/pii/S2212017313000145
Fazna korelacija je tehnika koju mo\zh emo koristiti za brzo pronala\zh enje mjesta preklapanja dviju sli\ch nih slika, \sh to nama upravo i treba. Mo\zh emo uzeti jedan blok piksela jednog frejma, te, koriste\cj i faznu korelaciju, brzo prona\cj i
gdje se taj blok piksela nalazi u drugom frejmu (ili barem neki drugi, veoma sli\ch an blok). Direktnom primjenom formule bismo dobili algoritam koji je zapravo veoma sli\ch an potpunoj pretrazi, koja je opisana u poglavlju o algoritmima
uparivanja blokova. Me\dj utim, mo\zh emo koristiti \textit{Brzu Fourierovu Transformaciju} (FFT) za zna\ch ajno ubrzanje. S obzirom na to da jedan korak algoritma uklju\ch uje izra\ch unavanje Hadamardovog proizvoda dvije matrice
(koji je jedino mogu\cj e izra\ch unati ako su matrice istih dimenzija), ovaj algoritam je ograni\ch en na tra\zh enje jednog ili vi\sh e mjesta poklapanja dva bloka piksela iste veli\ch ine. Tako\dj er, ako \zh elimo koristiti neki od FFT algoritama,
dimenzije blokova moraju biti stepeni broja dva.

Fazna korelacija je tehnika zasnovana na kori\sh tenju algoritma za brzu Fourierovu transformaciju za tra\zh enje mjesta poklapanja dva bloka veoma efikasno. Osim kori\sh tenja ve\cj ih blokova nego u prethodnim poglavljima,
tako\dj er \cj emo koristiti blokove koji se preklapaju, \sh to \cj e nam dati vi\sh e razli\ch itih kandidata za vektor pomaka jednog piksela. Od tih kandidata \cj emo izabrati vektor koji trenutni piksel uparuje sa najsli\ch nijim.
Pri podjeli frejmova na blokove, cilj nam je da svaki piksel originalnog frejma ima barem jedan blok u kojem se nalazi u kojem je njegov upareni piksel.
Ova metoda uparuje \ch itave blokove piksela sa drugim blokovima tako \sh to tra\zh i mjesta preklapanja u frekvencijskom domenu, \sh to je mogu\cj e uraditi u linearnom vremenu umjesto kvadratnom. S obzirom da je kompleksnost
FFT algoritama loglinearna, tra\zh enje fazne korelacije tako\dj er ima loglinearnu kompleksnost. Da bismo primijenili FFT algoritam, dimenzije blokova moraju biti stepeni broja 2. Preporu\ch ena veli\ch ina je 128x128 piksela.
Diskretna 2-D Fourierova transformacija je opisana sljede\cj om formulom:
\[ %http://www.di.univr.it/documenti/OccorrenzaIns/matdid/matdid027832.pdf
F(u,v)=\sum_{m=-\infty}^{+\infty}\sum_{n=-\infty}^{+\infty}f(m,n)e^{-2\pi i(um+vn)}
\]

Koraci algoritma\cite{phaseonly}:
\begin{enumerate}
\item Podijeli obje slike na preklapaju\cj e blokove kako je opisano
\item Za svaki piksel i svaki blok u kojem se taj piksel nalazi, izra\ch unaj 2D FFT piksela u tom bloku i u prethodnom i u narednom frejmu
\item Izra\ch unaj Hadamardov proizvod matrice dobivene iz prethodnog, i kompleksno konjugovane matrice dobivene iz narednog frejma
\item Normalizuj rezultat
\item Izra\ch unaj inverznu Fourierovu transformaciju
\item Vektor pomaka dobijemo kao koordinate maksimuma krajnje matrice
\end{enumerate}

Proces mo\zh e biti opisan sljede\cj om formulom:
\[
F^{-1}\Bigl\{\frac{G(k)H^*(k)}{|G(k)H^*(k)|}\Bigr\}
\]

Pri tome $F^{-1}$ predstavlja inverznu Fourierovu transformaciju, $G(k)$ i $H(k)$ su blok prethodnog i narednog frejma u frekvencijskom domenu, $^*$ je unarni operator kompleksne konjugacije, a proizvod matrica je zapravo Hadamardov
proizvod, odnosno jednostavno proizvod pojedina\ch nih elemenata na istim pozicijama u dvije matrice. 