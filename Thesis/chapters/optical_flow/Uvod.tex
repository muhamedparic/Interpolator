Sada \cj emo se po\ch eti baviti slo\zh enijim tehnikama interpolacije koje mogu davati mnogo bolje rezultate od linearne interpolacije. U idealnom slu\ch aju, rezultiraju\cj i frejmovi \cj e izgledati potpuno prirodno.
Postoji vi\sh e klasa algoritama koji su u praksi kori\sh teni, od kojih \cj emo mi u ovom poglavlju objasniti tri: Tehnike zasnovane na faznoj korelaciji, upore\dj ivanju blokova, i rje\sh avanju diferencijalnih jedna\ch ina. % Dodati jos diskretne ako uspijem
Zajedni\ch ko svim ovim algoritmima jeste da njihov izlaz ne\cj e biti novi interpolirani frejm, ve\cj\ \textit{polje opti\ch kog toka}\cite{wikioptflow}.
Cilj ovog poglavlja jeste da odgovori na sljede\cj a pitanja:
\begin{enumerate}
\item \SH ta su vektori pomaka?
\item \SH ta je opti\ch ki tok?
\item Kako mo\zh emo koristiti poznavanje opti\ch kog toka za kreiranje interpoliranih frejmova?
\item Kojim metodama mo\zh emo izra\ch unati opti\ch ki tok?
\end{enumerate}

Neka su zadana dva susjedna frejma, $A$ i $B$, izme\dj u kojih \zh elimo interpolirati novi frejm. Intuitivno gledaju\cj i, neki objekti frejma $A$ \cj e se tako\dj er nalaziti u frejmu $B$, samo na nekoj drugoj poziciji, blizu
svoje originalne. Drugim rije\ch ima, postoje vektori pomaka koji odgovaraju tim objektima, koji imaju dvije komponente i du\zh\ kojih su objekti prividno pomjereni izme\dj u frejmova $A$ i $B$.

Odre\dj ivanje vektora pomaka za objekte bi zahtjevalo tra\zh enje objekata u frejmovima, \sh to je izuzetno zahtjevno samo po sebi. Umjesto toga, svi algoritmi koje \cj emo objasniti \cj e pridru\zh ivati pojedina\ch ne
vektore pomaka svakom pikselu jednog od frejmova ($A$ ili $B$, neki algoritmi mogu odrediti koji je od ta dva slu\ch aja korisnije razmatrati). Algoritam \cj e jednostavno dobiti dva frejma, i za svaki piksel dati jedan
vektor, njegov vektor pomaka. Skup svih vektora pomaka piksela jednog frejma na drugi nazivamo poljem opti\ch kog toka, pri \ch emu opti\ch ki tok mo\zh emo definisati kao prividno kretanje objekata u video
sekvenci.

Najve\cj i dio posla pri generisanju interpoliranog frejma upravo jeste tra\zh enje kvalitetnog opti\ch kog toka. Nakon toga, proces je relativno jednostavan:
\begin{enumerate}
\item Svaki piksel pomjeriti du\zh\ svog vektora pomaka pomno\zh enog sa $\alpha$ (realan broj dobiven na isti na\ch in kao $\alpha$ u dijelu u linearnoj interpolaciji) te ga spasiti na tu poziciju.
\item Polja za koja nije prona\dj en niti jedan piksel popuniti kori\sh tenjem linearne interpolacije\cite{mefi}.
\end{enumerate}

\subsection{Kernel konvolucija} % http://web.pdx.edu/~jduh/courses/Archive/geog481w07/Students/Ludwig_ImageConvolution.pdf
Kernel konvolucija je tehnika \ch esto kori\sh tena za obradu slike, i ima veliki broj razli\ch itih primjena. Neke od njih uklju\ch uju detekciju ivica, izo\sh travanje, zamu\cj ivanje i druge razne filtere koji mogu biti primijenjeni
na slike. Ulazi kernel konvolucije su slika i sam kernel, dok je izlaz obra\dj ena slika. Kerneli funkcioni\sh u na samo jednom dvodimenzionalnom signalu, tako da \cj emo mi posmatrati samo intenzitete pojedinih piksela, 
ne njihove odvojene komponente (crvena, zelena i plava).

Kernel nije ni\sh ta drugo nego matrica realnih brojeva. Za na\sh e potrebe, dimenzije matrice \cj e morati biti neparni brojeva (vidjet \cj emo za\sh to je lak\sh e raditi sa takvim kernelima). Neka je zadana matrica intenziteta
piksela $A$, kernel $K$ i izlazna matrica $B$. Dimenzije izlazne matrice su iste kao dimenzije ulazne. Neka su dimenzije kernela (visina i \sh irina) $p$ i $q$. Svaki piksel izlazne matrice $B_{i,j}$ ra\ch unamo narednom formulom:

\[
B_{i,j}=\sum_{y=-p'}^{p'}\sum_{x=-q'}^{q'}A_{i+y,j+x}*K_{y+p'+1,x+q'+1} % POJEDNOSTATVITI FORMULU
\]
pri \ch emu su:
\[
p'=(p-1)/2
\]
\[
q'=(q-1)/2
\]
Zamislimo da smo postavili kernel iznad piksela sa koordinatama $i,j$, tako da element u sredini kernela ima upravo koordinate $i,j$. Ovdje vidimo za\sh to smo postavili ograni\ch enje da dimenzije kernela moraju biti neparni brojevi.
Zatim \cj emo pomno\zh iti svaki element kernela sa elementom matrice $A$ koji se nalazi ta\ch no ispod njega. Zbir svih tih proizvoda upisujemo u element $B_{i,j}$\cite{convolution}.

Kori\sh tenjem razli\ch itih kernela mo\zh emo dobiti razli\ch ite efekte. Ispod slijede primjeri nekih od \ch e\sh \cj e kori\sh tenih\cite{wikikernel}:
\vspace{10px} % Uljepsati sve
\[
\text{Detekcija ivica: } 
\begin{bmatrix}
-1 & -1 & -1 \\
-1 &  8 & -1 \\
-1 & -1 & -1
\end{bmatrix}
\]
\vspace{5px}
\[
\text{Zamu\cj ivanje: } 
\begin{bmatrix}
\frac{1}{9} & \frac{1}{9} & \frac{1}{9} \\
\frac{1}{9} &  \frac{1}{9} &\frac{1}{9} \\
\frac{1}{9} & \frac{1}{9} & \frac{1}{9}
\end{bmatrix}
\]
\vspace{5px}
\[
\text{Izo\sh travanje: } 
\begin{bmatrix}
0  & -1 &  0 \\
-1 &  5 & -1 \\
 0 & -1 &  0
\end{bmatrix}
\]

Neki od algoritama koje \cj emo objasniti koriste kernel konvoluciju, u opisu algoritama \cj emo navesti i konkretne kori\sh tene kernele.