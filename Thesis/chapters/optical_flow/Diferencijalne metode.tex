\subsection{Uvod}
Da bismo objasnili ove metode, moramo prvo pojasniti \textit{jedna\ch inu opti\ch kog toka}. Ova jedna\ch ina glasi\cite{wikioptflow, matlablk}:
\[
I_xu+I_yv+I_t=0
\]
ili
\[
\frac{\partial I}{\partial x}u+\frac{\partial I}{\partial y}v+\frac{\partial I}{\partial t}=0
\]

Jedan frejm, pri \ch emu samo posmatramo intenzitet piksela, a ne i njihove boje, mo\zh emo zamisliti kao realnu funkciju dvije realne promjenljive $I(x,y)$. Naravno, vrijednosti parametara
$x$ i $y$ moraju biti prirodni brojevi (ili nula), te \cj e sam rezultat funkcije biti cijeli broj u intervalu [0, 255], me\dj utim, uz ovakvo predstavljanje frejma mo\zh emo koristiti metode analize za
rje\sh avanje problema. Mo\zh emo \ch ak \ch itav video predstaviti kao jednu jedinu funkciju tri realne promjenljive $I(x,y,t)$, pri \ch emu $x$ i $y$ predstavljaju koordinate piksela u frejmu,
dok $t$ predstavlja redni broj samog frejma. Iz ovakve funkcije mo\zh emo izvesti jedna\ch inu opti\ch kog toka.

Za zadani piksel $I(x,y,t)$, na\sh\ cilj je prona\cj i $\Delta x,\Delta y,\Delta t$ takve da:
\[
I(x,y,t)=I(x+\Delta x,y+\Delta y,t+\Delta t)
\]

Zatim \cj emo $I(x,y,t)$ aproksimirati Taylorovim polinomom prvog reda:
\[
I(x+\Delta x,y+\Delta y,t+\Delta t)\approx I(x,y,t)+\frac{\partial I}{\partial x}\Delta x+\frac{\partial I}{\partial y}\Delta y+\frac{\partial I}{\partial t}\Delta t
\]

Koriste\cj i prethodne dvije jedna\ch ine, vidimo da je:
\[
I(x,y,t)\approx I(x,y,t)+\frac{\partial I}{\partial x}\Delta x+\frac{\partial I}{\partial y}\Delta y+\frac{\partial I}{\partial t}\Delta t
\]

Odnosno:
\[
\frac{\partial I}{\partial x}\Delta x+\frac{\partial I}{\partial y}\Delta y+\frac{\partial I}{\partial t}\Delta t \approx 0
\]

Ako obje strane jedna\ch ine podijelimo sa $\Delta t$ i uvedemo nove oznake:
\[
u=\frac{\Delta x}{\Delta t}
\]
\[
v=\frac{\Delta y}{\Delta t}
\]

Rezultat je upravo jedna\ch ina opti\ch kog toka (pri \ch emu smo znak $\approx$ zamijenili znakom $=$). Ako nas samo zanimaju pikseli u narednom frejmu (odnosno ako je $\Delta t=1$,
jedna\ch ina postaje:
\[
\frac{\partial I}{\partial x}\Delta x+\frac{\partial I}{\partial y}\Delta y+\frac{\partial I}{\partial t}=0
\] 

Vidimo da \ch ak ni ovu pojednostavljenu jedna\ch inu ne mo\zh emo jednozna\ch no rije\sh iti jer imamo dvije nepoznate ($\Delta x$ i $\Delta y$). Mi \cj emo u nastavku opisati dva algoritma koji
na dva razli\ch ita na\ch ina rje\sh avaju ovaj problem. Ako su nam za zadani piksel poznate vrijednosti $\Delta x$ i $\Delta y$, onda vektor $(\begin{smallmatrix}\Delta x &\Delta y\end{smallmatrix})$ predstavlja upravo tra\zh eni opti\ch ki tok,
koji mo\zh emo koristiti za kreiranje interpoliranih frejmova.

\subsection{Lucas-Kanade metoda}
\input{"chapters/optical_flow/Lucas-Kanade"}

\subsection{Horn-Schunck metoda}
\input{"chapters/optical_flow/Horn-Schunck"}