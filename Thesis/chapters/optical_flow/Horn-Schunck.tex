Horn-Schunck je metoda koja direktno ra\ch una sve vektore pomaka tako \sh to minimizira funkciju:
\[
E=\iint(I_xu+I_yv+I_t)^2dxdy+\alpha \iint\Bigl\{\frac{\partial u}{\partial x}^2+\frac{\partial u}{\partial y}^2+\frac{\partial v}{\partial x}^2+\frac{\partial v}{\partial y}^2\Bigr\}dxdy
\]

Dio funkcije pod prvim dvojnim integralom je dobiven direktno iz jedna\ch ine opti\ch kog toka. Minimizacijom prvog dijela \cj emo dobiti precizan opti\ch ki tok,
\ch iji su vektori pomaka sli\ch ni stvarnim. Dio funkcije pod drugim integralom predstavlja mjerilo glatko\cj e opti\ch kog toka. S obzirom da su $u$ i $v$ komponente
vektora pomaka, sabirci pod drugim integralom nam daju indikaciju prostornih promjena u opti\ch kom toku. Ve\cj e promjene zna\ch e i manje glatko polje
opti\ch kog toka, susjedni vektori pomaka se vi\sh e razlikuju. Minimizacijom ovog drugog dijela funkcije upravo pove\cj avamo glatko\cj u polja. Pove\cj avanjem
faktora $\alpha$ pove\cj avamo zna\ch aj glatko\cj e opti\ch kog toka, dok smanjivanjem pove\cj avamo zna\ch aj preciznosti.

$I_x$, $I_y$ i $I_t$ bismo mogli izra\ch unati kori\sh tenjem kernela koje smo naveli pri opisu Lucas-Kanade metode. Me\dj utim, Horn-Schunck metoda koristi
Sobel konvolucijski kernel za izra\ch unavanje prostornih izvoda:
\[
\begin{bmatrix}
-1 & 0 & 1 \\
-2 & 0 & 2 \\
-1 & 0 & 1
\end{bmatrix}
\]

i njegova transponovana forma. 

Nakon izra\ch unavanja $I_x$, $I_y$ i $I_t$, postavit \cj emo sve vektore pomaka na nula vektore, te ih postepeno pobolj\sh avati rekurzivnim formulama
koje predstavljaju sr\zh\ Horn-Schunck algoritma:
\[
u_{x,y}^{k+1}=\bar{u}_{x,y}^{k}-\frac{I_x(I_x\bar{u}_{x,y}^{k}+I_y\bar{v}_{x,y}^{k}+I_t)}{\alpha^2+I_x^2+I_y^2}
\]
\[
v_{x,y}^{k+1}=\bar{v}_{x,y}^{k}-\frac{I_x(I_x\bar{u}_{x,y}^{k}+I_y\bar{v}_{x,y}^{k}+I_t)}{\alpha^2+I_x^2+I_y^2}
\]

Pri tome, $k$ u superscriptu ozna\ch ava trenutnu vrijednost, a $k+1$ sljede\cj u vrijednost komponenti vektora opti\ch kog toka, dok $\bar{u_{x,y}}$ i $\bar{v_{x,y}}$
predstavljaju prosje\ch ne vrijednosti vektora pomaka za 4 piksela koji okru\zh uju piksel na poziciji $(x,y)$. U svakoj narednoj iteraciji \cj emo smanjiti vrijednost
funkcije.