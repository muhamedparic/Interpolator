Prva klasa algoritama koje \cj emo prou\ch avati kre\cj u od iste osnovne ideje: Podijeliti prvi frejm na blokove, te na\cj i vektor pomaka svakog bloka iz prvog frejma u drugi. Ovi algoritmi zna\ch ajno pojednostave problem tako \sh to
pretpostave da \cj e svi pikseli unutar jednog bloka imati iste vektore pomaka. Ovaj pristup mo\zh e dovesti do gre\sh aka, ali \ch ini izra\ch unavanje vektora pomaka jednostavnim i brzim.

Ako nam je cilj samo kreirati jedan novi frejm izme\dj u svaka dva postoje\cj a, svaki blok \cj emo pomjeriti du\zh\ pola izra\ch unatog vektora pomaka. Ako nam je cilj interpolirati dva frejma izme\dj u dva postoje\cj a, blok \cj emo 
pomjeriti du\zh\ jedne tre\cj ine vektora pomaka za prvi, i dvije tre\cj ine za drugi interpolirani frejm, itd. Cilj slijede\cj ih algoritama jeste uparivanje blokova prvog frejma sa blokom iste veli\ch ine u drugom frejmu. Me\dj utim,
postoji nekoliko pitanja na koja moramo odgovoriti prije nego \sh to mo\zh emo primijeniti ove algoritme:

\begin{itemize}
	\item Koju veli\ch inu bloka \cj emo koristiti?
	\item Koliki \cj e biti prozor pretrage, odnosno koliko \cj e se svaki blok mo\cj i maksimalno pomjeriti izme\dj u prvog i drugog frejma?
	\item Koji je kriterij sli\ch nosti dva bloka?
	\item Kako odrediti uspje\sh nost uparivanja?
\end{itemize}

U praksi se koriste blokovi veli\ch ine 16x16 piksela, te prozor pretrage veli\ch ine 30x30 piksela. To zna\ch i da pretpostavljamo da se izme\dj u dva susjedna frejma blokovi ne\cj e pomjeriti vi\sh e od 7 piksela u bilo kojem od 4
kardinalna smjera. To nam daje 225 mogu\cj ih lokacija za svaki blok. Naravno, ne postoji definitivna, optimalna veli\ch ina bloka ili prozora pretrage za sve slu\ch ajeve. Manje blokove je br\zh e uporediti, ali je njihov broj ve\cj i,
te je ve\cj a vjerovatno\cj a da \cj e dva bloka biti slu\ch ajno veoma sli\ch na. Ve\cj i prozor pretrage nam omogu\cj ava pronala\zh enje ispravnih vektora pomaka i u slu\ch aju kada se desi pomak ve\cj i od 7 piksela, ali
pove\cj ava vrijeme potrebno za izra\ch unavanje te, sli\ch no kao u slu\ch aju blokova, pove\cj anjem prozora pretrage se pove\cj ava i vjerovatno\cj a uparivanja dva bloka koji su sli\ch ni, ali zapravo ne
predstavljaju isti blok. U svim slijede\cj im algoritmima \cj emo koristiti blokove i prozore pretrage navedene veli\ch ine.

Sljede\cj e pitanje se ti\ch e kriterija sli\ch nosti dva bloka. Svaki blok je sastavljen od 256 piksela, koji se sastoje od 3 komponente: crvene, zelene, i plave. Svaka komponenta ima cjelobrojnu ja\ch inu u rasponu od 0 do 255, uklju\ch ivo.
Za upore\dj ivanje blokova ne\cj emo koristiti sve 3 komponente piksela, nego \cj emo izra\ch unati njihov intenzitet, koji se dobija kao aritmeti\ch ka sredina intenziteta crvene, zelene i plave komponente. Ovo se radi zato \sh to 
jednostavno ne bismo imali koristi od odvojenog upore\dj ivanja sve 3 komponente.
Naravno, postoji vjerovatno\cj a da \cj emo upariti dva piksela koji imaju sli\ch ne intenzitete, a ne i iste boje (npr. potpuno crven i potpuno zelen piksel imaju iste intenzitete), ali ra\ch unanjem vektora pomaka za sve 3 komponente
zasebno bismo \ch esto dobili 3 razli\ch ita vektora pomaka, \sh to nam ne daju nikakvu korisnu informaciju. Zbog toga je bolje samo uzimati u obzir intenzitete piksela.

Kori\sh teni kriteriji sli\ch nosti blokova su veoma jednostavni. Jedan je  \textit{Mean Absolute Difference (MAD)}, odnosno \textit{Srednja Apsolutna Razlika}. Ova mjera nije ni\sh ta drugo nego suma apsolutnih vrijednosti razlika
intenziteta odgovaraju\cj ih piksela u blokovima, podijeljena sa veli\ch inom bloka. Drugim rije\ch ima, zadana je formulom
$$
MAD = \frac{1}{N^2}\sum_{i=0}^{N-1}\sum_{j=0}^{N-1}|A_{ij}-B_{ij}|
$$

Pri \ch emu \textit{N} predstavlja visinu i \sh irinu bloka (u na\sh em slu\ch aju 16), dok $A_{ij}$ i $B_{ij}$ predstavljaju vrijednosti piksela na koordinatama $(i,j)$ (sa po\ch etkom u gornjem lijevom uglu bloka) prvog,
odnosno drugog razmatranog bloka. 

Druga, veoma sli\ch na mjera jeste \textit{Mean Squared Error (MSE)}, odnosno \textit{Srednji Kvadrat Gre\sh ke}. Umjesto uzimanja apsolutne vrijednosti razlika piksela, ova mjera kvadrira razlike piksela, \ch ime se vi\sh e
ka\zh njavaju ve\cj e razlike. \textit{MSE} je zadana formulom
$$
MSE = \frac{1}{N^2}\sum_{i=0}^{N-1}\sum_{j=0}^{N-1}(A_{ij}-B_{ij})^2
$$

Mi \cj emo ove funkcije ubudu\cj e zvati zajedni\ch kim imenom \textit{funkcije cijene}. Cilj svih algoritama u ovom poglavlju jeste minimizacija funkcije cijene, bilo \textit{MAD} ili \textit{MSE}.

Jo\sh\ jedno veoma va\zh no podru\ch je primjene algoritama uparivanja blokova (i zapravo podru\ch je gdje se najvi\sh e primjenjuju) jeste kompresija video sekvenci. O ovom poglavlju se obra\dj uju
algoritmi koji su kori\sh teni u standardima H.261, H.262 i H.263. U novijim standardima (pri \ch emu je najnoviji H.265, \ch ija je prva verzija iza\sh la 2013. godine) kori\sh teni su napredniji algoritmi, koji u
ovom radu ne\cj e biti obra\dj eni.

\subsection{PSNR}

Sada \cj emo se osvrnuti na neke od osnovnih algoritama uparivanja blokova. Pretpostavit \cj emo da su blokovi veli\ch ine 16x16 piksela, a prozori pretrage veli\ch ine 30x30 piksela, te da su svi pikseli monohromatski
(crno-bijeli). Od svih slijede\cj ih algoritama, samo prvi pronalazi optimalno uparivanje. Svi ostali ostali algoritmi su aproksimacije. Preciznije re\ch eno, postoji 225 mogu\cj ih lokacija gdje se po\ch tni blok mo\zh e nalaziti
u prozoru pretrage. Defini\sh imo matricu $C_{15x15}$, pri \ch emu svakom elementu matrice $C$ odgovara vrijednost funkcije cijene (\textit{MAD} ili \textit{MSE}) koju dobijemo ako postavimo blok na to mjesto u prozoru pretrage
(drugim rije\ch ima, elementu \textit{1,1} odgovara vrijednost funkcije cijene koju dobijemo ako blok postavimo u gornji lijevi ugao prozora pretrage, pomjeranjem bloka dobijamo druge vrijednosti matrice).
Aproksimativni algoritmi pretpostavljaju da ova matrica ima jednu najmanju (optimalnu) vrijednost, i da vrijednosti elemenata matrice monotono rastu kako se udaljavamo od ovog elementa. Kontinualni analogon ove osobine
bi bila funkcija 2 realne promjenljive koja ima samo jedan lokalni minimum, koji je ujedno i globalni minimum. Za funkciju koja ima ovu osobinu ka\zh emo da je \textit{unimodalna}. U slu\ch aju kad ova osobina postoji,
mo\zh emo prona\cj i minimalnu vrijednost matrice jednostavnim spu\sh tanjem po gradijentu, odnosno, po\ch ev\sh i od proizvoljnog po\ch etnog elementa, u svakom koraku pre\dj emo na bilo koji od manjih elemenata dok ne
do\dj emo do nekog koji je manji od svih svojih susjednih elemenata.

Naravno, ne postoji ni\sh ta da nam garantuje ovu osobinu, \sh to zna\ch i da \cj e aproksimativni algoritmi samo u rijetkim slu\ch ajevima prona\cj i optimalno rje\sh enje.
% Dodati dio o PSNR

\subsection{Potpuna pretraga}
Ovaj algoritam se zasniva na potpunom pretra\zh ivanju svih mogu\cj ih lokacija za blok, te pronala\zh enju lokacije koja daje najmanju vrijednost funkcije cijene. Postoji 225 mogu\cj ih lokacija za blok unutar prozora pretrage,
a za izra\ch unavanje vrijednosti funkcije cijene moramo uporediti $16 * 16 = 256$ piksela. To nam daje $225 * 256 = 57600$ upore\dj ivanja piksela po bloku. HD video (dimenzija 1280x720 piksela) sadr\zh i frejmove koji se
sastoje od 3600 blokova, \sh to nam ukupno daje preko 200 miliona upore\dj ivanja piksela za svaki frejm video sekvence. Tako da nije te\sh ko vidjeti za\sh to se ovaj algoritam ne koristi u praksi.

\subsection{Trostepena pretraga}
Trostepena pretraga se oslanja na osobinu unimodalnosti matrice vrijednosti funkcije cijene. Ovaj algoritam je ovdje kori\sh ten isklju\ch ivo kao uvod, jer ostvaruje veoma slabe rezultate u praksi. Algoritam je mnogo br\zh i
od potpune pretrage te koristi sli\ch ne tehnike kao bolji algoritmi koji \cj e biti obra\dj eni u narednim dijelovima.

Neka je zadan prozor pretrage veli\ch ine 30x30 piksela i blok veli\ch ine 16x16 piksela. Trostepena pretraga radi tako \sh to u svakom koraku izra\ch una funkciju cijene bloka na 9 lokacija u prozoru pretrage. Tih 9 lokacija su
trenutni centar i sve kombinacije lokacija koje dobijemo ako blok pomjerimo za \textit{-S, 0} ili \textit{S} piksela po x i y osama (postoji 9 kombinacija, od kojih je jedna \textit{(0, 0)}, \sh to predstavlja na\sh\ centar).
\textit{S} predstavlja trenutnu \textit{veli\ch inu koraka}. Nakon izra\ch unavanja funkcije cijene na svih 9 lokacija, izaberemo najbolju (tj. lokaciju koja daje najmanju vrijednost funkcije cijene), nju odredimo za novi centar, i 
smanjimo \textit{S} na jednu polovinu trenutne vrijednosti. U prvom koraku, \textit{S} \cj e imati vrijednost 4, u drugom 2, a u tre\cj em 1, dok \cj e centar u prvom koraku predstavljati stvarni centar prozora pretrage. 
S obzirom da se na po\ch etku blok nalazi 7 piksela daleko od ivica prozora pretrage, a u 3 koraka (sa pomakom od 4, 2 i 1 pikselom) se mogu\cj e pomjeriti najvi\sh e ta\ch no 7 piksela, blok mo\zh e zauzeti bilo koju 
poziciju u prozoru pretrage.

U svakom od 3 koraka, imamo 9 razli\ch itih mogu\cj ih pozicija bloka, tako da izgleda da moramo izra\ch unati funkciju cijene bloka 27 puta. Me\dj utim, centar u drugom i tre\cj em koraku je ve\cj\ izra\ch unat u prethodnom
koraku, tako da je zapravo potrebno izra\ch unati samo 25 razli\ch itih funkcija pretrage, \sh to je 9 puta manje od 225 kod potpune pretrage, \sh to naravno predstavlja veoma zna\ch ajno ubrzanje. Me\dj utim, kao \sh to je
prije re\ch eno, ovaj algoritam generalno ne izra\ch unava zadovoljavaju\cj e vektore pomaka, tako da se u praksi ne koristi.
%Slika rada algoritma

\subsection{\CH etverostepena pretraga}
\CH etverostepena pretraga radi na istom principu kao trostepena, samo sa druga\ch ijim izborom veli\ch ine koraka. U prvom koraku $S$ iznosi 2, odnosno lokacije koje pretra\zh ujemo \cj e biti $(0, 0)$, $(0, 2)$, $(2, 0)$, $(2, 2)$,
$(-2, 0)$, $(0, -2)$, $(-2, -2)$, $(2, -2)$ i $(-2, 2)$. U slu\ch aju da lokacija $(0, 0)$ daje najmanju vrijednost funkcije cijene, prelazimo na \ch etvrti korak. Ovo vrijedi i u drugom i tre\cj em koraku, za trenutni centar pretrage. Drugi i
tre\cj i korak tako\dj er imaju veli\ch inu koraka $S=2$, dok za \ch etvrti korak vrijedi $S=1$. Ukoliko pro\dj emo kroz sva \ch etiri koraka, vidimo da je maksimalni pomak jednak $2+2+2+1=7$.

Ovaj pristup zahtijeva od 17 do 27 upore\dj ivanja blokova. Razlog ovome jeste \sh to je veli\ch ina koraka u prva 3 koraka ista, \sh to zna\ch i da \cj e se zna\ch ajan broj lokacija poklapati, te ih ne moramo ra\ch unati ponovno.
Iako je manje konsistentna od trostepene pretrage (koja uvijek zahtijeva 25 upore\dj ivanja blokova), \ch etverostepena pretraga \cj e u ve\cj ini slu\ch ajeva biti br\zh a. Druga prednost je \sh to \ch etverostepena pretraga bolje
detektuje male vektore pomaka (1 do 2 piksela u jednom smjeru), zato \sh to je po\ch etni korak manji ($S=2$ u odnosu na $S=4$), kao i zato \sh to u slu\ch aju da je najbolja lokacija u centru, automatski prelazimo u \ch etvrti korak,
koji mo\zh e samo jos podesiti vektor pomaka za 1 piksel u bilo kojem smjeru.

\subsection{Dijamantna pretraga}
Dijamantna pretraga je sli\ch na \ch etverostepenoj pretrazi, uz dvije modifikacije. Prva je da oblik koji prave lokacije koje pretra\zh ujemo nije vi\sh e kvadrat \ch ije su stranice paralelne sa koordinatnim osama, nego kvadrat 
rotiran za 45\degree (\ch iji oblik podsje\cj a na dijamant ili romb), a druga da ne postoji gornja granica na maksimalan broj koraka. Naravno, algoritam \cj e uvijek terminirati jer postoji 225 mogu\cj ih lokacija, a funkcija cijene 
za svaku lokaciju se ra\ch una samo jednom,
druga razlika samo zna\ch i da algoritam ne\cj e automatski pre\cj i na manju veli\ch inu koraka nakon nekog broja koraka, nego se to radi samo ako, od 9 lokacija koje razmatramo u trenutnom koraku, najbolji rezultat daje
trenutni centar. Do posljednjeg koraka, veli\ch ina koraka $S$ iznosi 2, \sh to u prvom koraku daje sljede\cj e lokacije koje trebamo pretra\zh iti: $(0, 0)$, $(2, 0)$, $(0, 2)$, $(-2, 0)$, $(0, -2)$, $(1, 1)$, $(1, -1)$, $(-1, 1)$ i $(-1, -1)$.
U posljednjem koraku, veli\ch ina koraka $S$ iznosi 1, \sh to zna\ch i da \cj emo posmatrati samo trenutni centar i 4 lokacije direktno iznad, ispod, lijevo i desno od trenutnog centra (pomjerene po jedan piksel u kardinalnim
smjerovima).

Garanciju da \cj e algoritam eventualno pre\cj i na posljednji korak nam daje \ch injenica da ovaj algoritam (i oba prethodna) uvijek prelazi na lokaciju koja daje bolju vrijednost funkcije cijene (ili ostaje na trenutnoj lokaciji). 
S obzirom da postoji kona\ch an (225) broj lokacija, a nikad ne prelazimo na lokaciju koja nam daje ve\cj u vrijednost funkcije cijene, eventualno \cj emo do\cj i do lokalnog minimuma (koji je u idealnom slu\ch aju i globalni
minimum).

\subsection{Adaptive Rood Pattern Search - ARPS}
Algoritam koji daje najbolje rezultate koji \cj emo opisati u ovom poglavlju je ARPS. Za razliku od do sada opisanih algoritama, ARPS uzima u obzir ne samo vrijednosti piksela bloka i prozora pretrage, nego i prethodno
izra\ch unati vektor pretrage. Konkretno, da bismo koristili ARPS, potreban nam je jedan vektor pomaka za koji o\ch ekujemo da \cj e vjerovatno biti sli\ch an stvarnom vektoru pomaka trenutnog bloka. Mi \cj emo koristiti
vektor pomaka bloka koji se nalazi lijevo od trenutnog u tu svrhu. Ako se trenutni blok nalazi uz lijevu ivicu frejma (i ne postoji blok lijevo od njega), onda \cj emo koristiti vektor pomaka bloka iznad trenutnog. A u krajnjem
slu\ch aju, ako ra\ch unamo vektor pomaka za prvi blok (u gornjem lijevom uglu frejma), mo\zh emo koristiti potpunu pretragu koja zagarantovano daje optimalno uparivanje (uticaj na performanse nije zna\ch ajan jer
potpunu pretragu vr\sh imo samo jednom za jedan frejm).

Neka prethodni vektor pomaka iznosi $(x, y)$. Uvest \cj emo pomo\cj nu promjenljivu $h=max(|x|,|y|)$. U prvom koraku \cj emo izra\ch unati funkciju cijene na 6 lokacija: (0, 0), (h, 0), (0, h), (-h, 0), (0, -h), (x, y). Odnosno,
pretra\zh ujemo lokaciju (0, 0) (jer su nula vektori pomaka \ch esti), vektor pomaka od kojeg smo krenuli (x, y), te 4 lokacije koje su isto daleko od (0, 0) po tzv. \textit{Manhattan udaljenosti}. Treba uzeti u obzir da je mogu\cj e
da je $x=0$ ili $y=0$, u kojem slu\ch aju je potrebno pretra\zh iti samo 5 lokacija (2 se poklapaju). Ako je $x=y=0$, onda je rije\ch\ o samo jednoj lokaciji, i mo\zh emo odmah pre\cj i na drugi korak algoritma.

Drugi korak ARPS algoritma je isti kao drugi korak dijamantne pretrage. Od pretra\zh enih lokacija izaberemo najbolju, zatim izra\ch unavamo 4 lokacije direktno iznad, ispod, lijevo i desno od trenutne najbolje, te prelazimo
na najbolju od te 4. Algoritam se zavr\sh ava kada je trenutna lokacija bolje od 4 koje je okru\zh uju. Kao i uvijek, mo\zh emo ubrzati algoritam tako \sh to ne ra\ch unamo opet funkciju cijene za lokacije koje smo ve\cj\ 
obradili.

% Izbaciti iz ovog dijela
%\section{Kros-korelacija} % Sa MATLAB-ove stranice za xcorr2
%Svi do sada obja\sh njeni algoritmi se zasnivaju na direktnom upore\dj ivanju piksela. Kros-korelacija, s druge strane, je metoda sli\ch na konvoluciji, te tako\dj er radi, u osnovi, direktno upore\dj ivanje piksela 2 bloka u
%cilju pronala\zh enja lokacije prvog (manjeg) bloka unutar drugog (ve\cj eg). Ina\ch e, treba napomenuti da se radi o diskretnoj kros-korelaciji, jer je skup piksela nad kojim se vr\sh i ova operacija po svojoj prirodi
%diskretan. Kao \sh to \cj emo vidjeti, kros-korelacija u su\sh tini radi potpunu pretragu, te implementirana direktno ima isti problem izuzetno visokog vremena izvr\sh avanja. Me\dj utim, kros-korelaciju je mogu\cj e
%izra\ch unati kori\sh tenjem brze Fourierove transformacije (\textit{FFT}), \sh to \cj e zna\ch ajno ubrzati pretragu. Kros korelacije se naj\ch e\sh \cj e ozna\ch zava simbolom *. U nastavku \cj emo objasniti sam algoritam, 
%a nakon toga \cj e biti govora o njegovim performansama.

%Jednodimenzionalna diskretna kros korelacija nizova \textit{f} i \textit{g} se ra\ch una formulom: 
%$$
%(f * g) [n] := \sum_{m=-\infty}^{\infty}f^{*}[m]g[m+n]
%$$