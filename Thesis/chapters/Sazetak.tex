U radu su obja\sh njeni klju\ch ni pojmovi vezani za interpolaciju video sekvenci, poput frejmova, vektora pomaka i opti\ch kog toka. Nakon toga,
obja\sh njen je na\ch in na koji mo\zh emo interpolirati frejmove ako je opti\ch ki tok poznat, te problemi koji nastaju i kako ih rije\sh iti. U glavnom dijelu rada su zatim
obja\sh njeni algoritmi pronala\zh enja opti\ch kog toka, koji mogu raditi na principu uparivanja blokova, fazne korelacije ili putem metoda zasnovanim na diferencijalnim
jedna\ch inama. Prostom primjenom ovih algoritama \cj e interpolirani frejmovi biti veoma slabe kvalitete, tako da su tako\dj er opisane metode za popravljanje gre\sh aka
i artefakta. U posljednjem poglavlju je opisana implementacija interpolatora u programskom jeziku C++ uz kori\sh tenje OpenCV biblioteke.